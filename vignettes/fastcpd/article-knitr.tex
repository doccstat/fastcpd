% \citep[][Chapter~7.4]{Venables+Ripley:2002}
% knitr::Sweave2knitr("article.Rnw"); knitr::knit("article-knitr.Rnw")
\documentclass[article]{jss}\usepackage[]{graphicx}\usepackage[]{xcolor}
% maxwidth is the original width if it is less than linewidth
% otherwise use linewidth (to make sure the graphics do not exceed the margin)
\makeatletter
\def\maxwidth{ %
  \ifdim\Gin@nat@width>\linewidth
    \linewidth
  \else
    \Gin@nat@width
  \fi
}
\makeatother

\definecolor{fgcolor}{rgb}{0.345, 0.345, 0.345}
\newcommand{\hlnum}[1]{\textcolor[rgb]{0.686,0.059,0.569}{#1}}%
\newcommand{\hlstr}[1]{\textcolor[rgb]{0.192,0.494,0.8}{#1}}%
\newcommand{\hlcom}[1]{\textcolor[rgb]{0.678,0.584,0.686}{\textit{#1}}}%
\newcommand{\hlopt}[1]{\textcolor[rgb]{0,0,0}{#1}}%
\newcommand{\hlstd}[1]{\textcolor[rgb]{0.345,0.345,0.345}{#1}}%
\newcommand{\hlkwa}[1]{\textcolor[rgb]{0.161,0.373,0.58}{\textbf{#1}}}%
\newcommand{\hlkwb}[1]{\textcolor[rgb]{0.69,0.353,0.396}{#1}}%
\newcommand{\hlkwc}[1]{\textcolor[rgb]{0.333,0.667,0.333}{#1}}%
\newcommand{\hlkwd}[1]{\textcolor[rgb]{0.737,0.353,0.396}{\textbf{#1}}}%
\let\hlipl\hlkwb

\usepackage{framed}
\makeatletter
\newenvironment{kframe}{%
 \def\at@end@of@kframe{}%
 \ifinner\ifhmode%
  \def\at@end@of@kframe{\end{minipage}}%
  \begin{minipage}{\columnwidth}%
 \fi\fi%
 \def\FrameCommand##1{\hskip\@totalleftmargin \hskip-\fboxsep
 \colorbox{shadecolor}{##1}\hskip-\fboxsep
     % There is no \\@totalrightmargin, so:
     \hskip-\linewidth \hskip-\@totalleftmargin \hskip\columnwidth}%
 \MakeFramed {\advance\hsize-\width
   \@totalleftmargin\z@ \linewidth\hsize
   \@setminipage}}%
 {\par\unskip\endMakeFramed%
 \at@end@of@kframe}
\makeatother

\definecolor{shadecolor}{rgb}{.97, .97, .97}
\definecolor{messagecolor}{rgb}{0, 0, 0}
\definecolor{warningcolor}{rgb}{1, 0, 1}
\definecolor{errorcolor}{rgb}{1, 0, 0}
\newenvironment{knitrout}{}{} % an empty environment to be redefined in TeX

\usepackage{alltt}

%% -- LaTeX packages and custom commands ---------------------------------------

%% recommended packages
\usepackage{orcidlink,thumbpdf,lmodern}
\usepackage{mathtools}
\usepackage{booktabs}
\usepackage{multirow}

\usepackage{geometry}
\usepackage{pdflscape}

%% another package (only for this demo article)
\usepackage{framed}

%% new custom commands
\newcommand{\class}[1]{`\code{#1}'}
\newcommand{\fct}[1]{\code{#1()}}

%% For Sweave-based articles about R packages:
%% need no \usepackage{Sweave}



\begin{knitrout}
\definecolor{shadecolor}{rgb}{0.969, 0.969, 0.969}\color{fgcolor}\begin{kframe}


{\ttfamily\noindent\itshape\color{messagecolor}{\#\# Loading required package: bigmemory}}

{\ttfamily\noindent\itshape\color{messagecolor}{\#\# Loading required package: foreach}}

{\ttfamily\noindent\itshape\color{messagecolor}{\#\# Loading required package: rngtools}}\end{kframe}
\end{knitrout}


%% -- Article metainformation (author, title, ...) -----------------------------

%% - \author{} with primary affiliation (and optionally ORCID link)
%% - \Plainauthor{} without affiliations
%% - Separate authors by \And or \AND (in \author) or by comma (in \Plainauthor).
%% - \AND starts a new line, \And does not.
\author{First Author~\orcidlink{0009-0006-2493-0853}\\Texas A\&M University
   \And Second Author~\orcidlink{0009-0006-2493-0853}\\Texas A\&M University}
\Plainauthor{First Author, Second Author}

%% - \title{} in title case
%% - \Plaintitle{} without LaTeX markup (if any)
%% - \Shorttitle{} with LaTeX markup (if any), used as running title
\title{\pkg{fastcpd}: Fast Change Point Detection in \proglang{R}}
\Plaintitle{fastcpd: Fast Change Point Detection in R}
\Shorttitle{Fast Change Point Detection in \proglang{R}}

%% - \Abstract{} almost as usual
\Abstract{
  TBD
}

%% - \Keywords{} with LaTeX markup, at least one required
%% - \Plainkeywords{} without LaTeX markup (if necessary)
%% - Should be comma-separated and in sentence case.
\Keywords{change point detection, gradient descent, quasi-newton, incremental gradient, \proglang{R}}
\Plainkeywords{change point detection, gradient descent, quasi-newton, incremental gradient, R}

%% - \Address{} of at least one author
%% - May contain multiple affiliations for each author
%%   (in extra lines, separated by \emph{and}\\).
%% - May contain multiple authors for the same affiliation
%%   (in the same first line, separated by comma).
\Address{
  Contact Author\\
  Journal of Statistical Software\\
  \emph{and}\\
  Department of Statistics\\
  Faculty of Economics and Statistics\\
  Universit\"at Innsbruck\\
  Universit\"atsstr.~15\\
  6020 Innsbruck, Austria\\
  E-mail: \email{Achim.Zeileis@R-project.org}\\
  URL: \url{https://www.zeileis.org/}
}
\IfFileExists{upquote.sty}{\usepackage{upquote}}{}
\begin{document}


%% -- Introduction -------------------------------------------------------------

\section[Introduction: Fast change point detection in R]{Introduction: Fast change point detection in \proglang{R}} \label{sec:intro}

\pkg{fastcpd} (\textbf{FAST} \textbf{C}hange \textbf{P}oint \textbf{D}etection)
is an R package for fast offline change-point detection.
Change-point detection is an important problem in many fields, including signal
processing, finance, and biology. Given a sequence of data points, a
change-point detection algorithm identifies points where the underlying
statistical properties of the data change. \pkg{fastcpd} implements a novel
approach for change-point detection that is orders of magnitude faster than
existing methods, without sacrificing accuracy.

The \pkg{fastcpd} method is based on a sequential optimization algorithm that
uses gradient descent to efficiently search for change-points. The algorithm is
designed to minimize a cost function that captures the likelihood of observing
the data given a particular segmentation. The key idea is to update the cost
function using information from previous steps, rather than re-optimizing the
objective function at each step. This allows \pkg{fastcpd} to quickly identify
change-points in long sequences of data, even when the cost function involves
solving a non-trivial optimization problem.

\pkg{fastcpd} supports change-point detection in a variety of settings,
including generalized linear models and penalized regression. The package
provides a simple and intuitive interface for users to specify their data and
desired parameters, and produces output in a convenient format for further
analysis. In addition, \pkg{fastcpd} includes a suite of visualization tools for
exploring the results of the change-point detection algorithm.

We believe that \pkg{fastcpd} will be a valuable tool for researchers and
practitioners working with change-point detection problems. By providing fast
and accurate change-point detection in a user-friendly package, \pkg{fastcpd}
will enable users to more easily analyze and understand complex datasets.

In \proglang{R} \citep{R}, \cite{killick2014changepoint} has implemented
\pkg{changepoint} to detect change points with a linear computational cost.
Functions including \fct{cpt.mean}, \fct{cpt.var}, \fct{PELT} etc are provided.
The manuscript that this document is based on \citep{zhang2022sequential}
then introduced sequential gradient descent and quasi-Newton's method.

%% -- Manuscript ---------------------------------------------------------------

\section{Models and software} \label{sec:models}

Logistic regression, Poisson regression, linear regression, and penalized
linear regression have already been implemented in \pkg{fastcpd} package. Some
specifications are listed in Table~\ref{tab:models}. The signature of the main
function \fct{fastcpd} is
%
\begin{Code}
fastcpd(
  data,
  beta,
  segment_count = 10,
  trim = 0.025,
  momentum_coef = 0,
  sgd_k = 3,
  family = NULL,
  epsilon = 1e-10,
  min_prob = 10^10,
  winsorise_minval = -20,
  winsorise_maxval = 20,
  p = NULL,
  cost = negative_log_likelihood,
  cost_gradient = cost_update_gradient,
  cost_hessian = cost_update_hessian
)
\end{Code}
%
where each parameters have the following usages:
\begin{itemize}
  \item \code{data}: A data frame containing the data to be segmented where each
    row denotes each data point. In regression settings, the first column is the
    response variable while the rest are covariates.
  \item \code{beta}: Initial cost value specified in Algorithm~1 in
    \cite{zhang2022sequential}.
  \item \code{segment_count}: Number of segments for initial guess. If not
    specified, the initial guess on the number of segments is 10.
  \item \code{trim} Trimming for the boundary change points so that a change
    point close to the boundary will not be counted as a change point. This
    parameter also specifies the minimum distance between two change points. If
    several change points have mutual distances smaller than
    \code{trim * nrow(data)}, those change points will be merged into one single
    change point.
  \item \code{momentum_coef} Momentum coefficient to be applied to each update.
    This parameter is used when the loss function is bad-shaped so that
    maintaining a momentum from previous update is desired. Default value is 0,
    meaning the algorithm doesn't maintain a momentum by default.
  \item \code{sgd_k} Number of epochs in for each update whenever the algorithm
    takes a new data point, in the sense that the data are analyzed sequentially
    according to the nature of Pruned Exact Linear Time (PELT) algorithm
    \citep{killick2012optimal}.
  \item \code{family} Family of the model. Can be ``\code{binomial}'', ``\code{poisson}'',
    ``\code{lasso}'', ``\code{gaussian}'' or ``\code{custom}''. For simplicity,
    user can also omit this parameter, indicating that they will be using their
    own cost functions. If specified as ``\code{custom}'' or ``\code{NULL}'', the user must
    specify the cost function, gradient and corresponding Hessian matrix.
    Hessian is preferred when the user want to specify their own cost function,
    but not analytically available, the user should provide a single number
    (diagonal matrix) to replace the Hessian matrix. Should be left as
    \code{NULL} if the user would like to use their own cost functions.
  \item \code{epsilon} Epsilon to avoid numerical issues. Only used for Logistic
    Regression and Poisson Regression.
  \item \code{min_prob} Minimum probability to avoid numerical issues. Only used
    for Poisson Regression.
  \item \code{winsorise_minval} Minimum value for the parameter in Poisson
    Regression to be winsorised.
  \item \code{winsorise_maxval} Maximum value for the parameter in Poisson
    Regression to be winsorised.
  \item \code{lambda} Lambda for L1 regularization. Only used in ``lasso''.
  \item \code{cost} Cost function to be used. This and the following two
    parameters should not be specified at the same time with \code{family}.
    If not specified, the default is the negative log-likelihood for the
    corresponding family.
  \item \code{cost_gradient} Gradient function for the custom cost function.
  \item \code{cost_hessian} Hessian function for the custom cost function.
\end{itemize}
Return of the function is a list containing two elements:
\begin{itemize}
  \item \code{cp_set}: A vector containing the change points.
  \item \code{cost_value}: Values of the cost function for each data segments
    separated by the change points.
\end{itemize}

A \class{fastcpd} object is returned by the function. This object can be used
to plot the change points and the cost function values. Compatible functions
include \fct{plot}, \fct{print} and \fct{summary}.

\begin{table}[t!]
\centering
\begin{tabular}{llp{10cm}}
\hline
Model               & \code{family}   & Description \\ \hline
Linear Regression   & \code{gaussian} & Will be using pre-defined cost functions,
                                        details can be found in
                                        Section~\ref{sec:linear model} \\
Logistic Regression & \code{binomial} & Will be using pre-defined cost functions,
                                        details can be found in
                                        Section~\ref{sec:logistic_regression} \\
Poisson Regression  & \code{poisson}  & Will be using pre-defined cost functions,
                                        details can be found in
                                        Section~\ref{sec:poisson} \\
LASSO               & \code{lasso}    & Will be using pre-defined cost functions,
                                        details can be found in
                                        Section~\ref{sec:linear model} \\ \hline
User-defined Model  & \code{custom}   & If \code{family} is specified as ``custom''
                                        or \code{NULL}, \code{cost}, \code{cost_gradient}
                                        and \code{cost_hessian} must be provided. \\
Huber Regression    & \code{custom}   & Instead of providing Huber Regression
                                        as a built-in model, we provide an example
                                        of how to use \code{custom} model. Details
                                        can be found in Section~\ref{sec:huber} \\
Quantile Regression & \code{custom}   & Instead of providing Quantile Regression
                                        as a built-in model, we provide an example
                                        of how to use \code{custom} model. Details
                                        can be found in Section~\ref{sec:quantile} \\
Mean Shift          & \code{custom}   & Instead of providing Quantile Regression
                                        as a built-in model, we provide an example
                                        of how to use \code{custom} model. Details
                                        can be found in Section~\ref{sec:mean shift model} \\ \hline
\end{tabular}
\caption{\label{tab:models} All the models that have been implemented in the
package, including an empty model so that the users are able to provide their
own.}
\end{table}

\section{Changes in linear models} \label{sec:linear model}



\begin{knitrout}
\definecolor{shadecolor}{rgb}{0.969, 0.969, 0.969}\color{fgcolor}\begin{kframe}
\begin{alltt}
\hlkwd{summary}\hlstd{(linear_regression_result)}
\end{alltt}
\begin{verbatim}
## 
## Call:
## fastcpd(data = data, beta = beta, k = function(x) 0, family = "gaussian", 
##     epsilon = 1e-05)
## 
## [1] "Residuals:"
##        Min         1Q     Median         3Q        Max 
## -1.9500129 -0.4839276  0.0068760  0.4708422  1.8456402 
## 
## Change points:
## [1]  76 150 225
\end{verbatim}
\begin{alltt}
\hlkwd{plot}\hlstd{(linear_regression_result)}
\end{alltt}
\end{kframe}
\includegraphics[width=\maxwidth]{figure/linear_regression_plot-1} 
\end{knitrout}

\newgeometry{margin=1cm}
\begin{landscape}

\begin{table}[t!]
  \centering
  \begin{tabular}{@{}cccccccccc@{}}
  \toprule
                          & \multicolumn{3}{c}{p=1}                                            & \multicolumn{3}{c}{p=3}                                            & \multicolumn{3}{c}{p=5}                                            \\
                          & n=300                & n=600                & n=1500               & n=300                & n=600                & n=1500               & n=300                & n=600                & n=1500               \\ \midrule
  \multirow{4}{*}{cpc=0} & 1.00, 1.48s & 1.00, 3.88s & 1.00, 12.12s & 1.00, 1.59s & 1.00, 3.36s & 1.00, 9.72s & 1.00, 1.81s & 1.00, 3.87s & 1.00, 12.16s \\
                          & 1.00, 103.61s & 1.00, 341.06s & 1.00, 1178.56s & 1.00, 61.17s & 1.00, 226.55s & 1.00, 1283.73s & 1.00, 96.40s & 1.00, 282.26s & 1.00, 913.02s \\
                          & 1.00, 156.23s & 1.00, 481.71s & 1.00, 1376.99s & 1.00, 78.88s & 1.00, 228.14s & 1.00, 3373.77s & 1.00, 109.70s & 1.00, 346.53s & 1.00, 3554.79s \\
                          & 0.90, 40.71s & 1.00, 194.13s & 1.00, 1160.40s & 1.00, 37.18s & 0.90, 168.97s & 0.91, 603.55s & 0.56, 28.89s & 0.40, 93.09s & 0.77, 473.10s \\ \midrule
  \multirow{4}{*}{cpc=1} & 0.98, 2.40s & 0.98, 6.25s & 0.98, 14.13s & 0.98, 1.72s & 0.99, 4.53s & 0.99, 16.93s & 0.99, 2.18s & 0.99, 5.41s & 1.00, 14.97s \\
                          & 0.91, 93.22s & 0.98, 428.46s & 1.00, 2755.46s & 0.97, 92.85s & 0.98, 348.92s & 1.00, 2078.98s & 1.00, 69.20s & 1.00, 288.67s & 1.00, 1975.72s \\
                          & 0.96, 186.53s & 0.99, 931.36s & 0.99, 4333.45s & 0.99, 158.12s & 0.99, 602.64s & 1.00, 4723.81s & 0.98, 210.51s & 1.00, 440.30s & 1.00, 2915.44s \\
                          & 0.96, 31.05s & 1.00, 110.85s & 0.97, 650.02s & 0.92, 19.16s & 0.97, 87.13s & 0.97, 574.22s & 0.98, 21.14s & 0.88, 73.06s & 0.92, 417.99s \\ \midrule
  \multirow{4}{*}{cpc=3} & 0.75, 2.58s & 0.99, 7.00s & 0.99, 16.35s & 0.83, 2.01s & 0.99, 4.20s & 1.00, 10.48s & 0.84, 2.14s & 0.99, 4.95s & 1.00, 14.83s \\
                          & 0.83, 116.19s & 0.97, 586.45s & 0.99, 2609.28s & 0.82, 74.45s & 0.99, 218.70s & 1.00, 1940.88s & 0.92, 77.67s & 0.99, 323.52s & 1.00, 1697.94s \\
                          & 0.82, 227.43s & 0.97, 903.17s & 0.99, 3847.89s & 0.81, 199.59s & 0.99, 517.29s & 1.00, 2144.98s & 0.92, 113.10s & 0.92, 618.55s & 1.00, 3090.91s \\
                          & 0.92, 23.83s & 0.96, 70.75s & 0.98, 373.69s & 0.98, 14.70s & 0.98, 45.15s & 1.00, 289.93s & 0.95, 10.81s & 0.94, 44.93s & 0.98, 242.29s \\ \midrule
  \multirow{4}{*}{cpc=5} & 0.67, 2.82s & 0.82, 7.87s & 0.99, 15.63s & 0.55, 2.32s & 0.88, 7.17s & 0.99, 16.55s & 0.75, 1.86s & 0.96, 4.67s & 1.00, 10.74s \\
                          & 0.79, 111.85s & 0.83, 455.71s & 0.99, 2270.08s & 0.83, 46.30s & 0.93, 311.22s & 0.99, 1719.32s & 0.79, 72.71s & 1.00, 115.01s & 1.00, 1357.81s \\
                          & 0.52, 200.35s & 0.89, 848.59s & 0.99, 2710.85s & 0.94, 92.61s & 0.99, 294.48s & 0.99, 2647.00s & 0.91, 138.84s & 1.00, 366.56s & 0.99, 2266.15s \\
                          & 0.87, 17.31s & 0.97, 59.14s & 0.99, 225.48s & 0.97, 10.93s & 0.99, 42.71s & 0.99, 203.91s & 0.96, 9.39s & 0.98, 30.35s & 0.98, 150.41s \\ \bottomrule
  \end{tabular}
  \caption{\label{tab:linear regression comparison} Comparison of the algorithms for the linear regression model in mean.}
\end{table}

\end{landscape}
\restoregeometry

\section{Changes in generalized linear models: Logistic regression} \label{sec:logistic_regression}

We first consider the change-point detection problem in the generalized linear
models (GLM). Suppose we have a data set containing a set of
predictors/covariates and corresponding response variables, i.e. each data point
\code{data[i, ]} contains a set of predictors/covariates \code{data[i, -1]} and
a response \code{data[i, 1]}. The reason we set the first column to be the
response variables are that in \proglang{R}, we can use \code{-1} to denote all
columns other than the first column without the need to consider number of
columns altogether.

Suppose the response variables follows a binomial distribution with specifics
defined as
%
\begin{equation} \label{eq:logistic_regression}
y_i \sim \mathrm{Bernoulli}\left(\frac{1}{1 + e^{- x_i^\top \theta_i}}\right),
\quad x_i \sim \mathcal{N}_p(0, \Sigma)\ \mathrm{with}\ \Sigma =
(0.9^{\lvert i - j \rvert})_{p \times p}, \quad 1 \le i \le n,
\end{equation}
%
where $\{y_i\}$ is the response variable, $\{x_i\}$ is the covariate vector,
$\Sigma$ is the covariance matrix for the sampling of $\{x_i\}$, $p$ is the
number of predictors, $n$ is the total number of data points.

We now use the \pkg{fastcpd} package to detect the change points in the data set
\code{data}.
%
\begin{knitrout}
\definecolor{shadecolor}{rgb}{0.969, 0.969, 0.969}\color{fgcolor}\begin{kframe}


{\ttfamily\noindent\color{warningcolor}{\#\# Warning: fit\_glm: fitted probabilities numerically 0 or 1 occurred}}

{\ttfamily\noindent\color{warningcolor}{\#\# Warning: fit\_glm: fitted probabilities numerically 0 or 1 occurred}}

{\ttfamily\noindent\color{warningcolor}{\#\# Warning: fit\_glm: fitted probabilities numerically 0 or 1 occurred}}

{\ttfamily\noindent\color{warningcolor}{\#\# Warning: fit\_glm: fitted probabilities numerically 0 or 1 occurred}}

{\ttfamily\noindent\color{warningcolor}{\#\# Warning: fit\_glm: fitted probabilities numerically 0 or 1 occurred}}

{\ttfamily\noindent\color{warningcolor}{\#\# Warning: fit\_glm: fitted probabilities numerically 0 or 1 occurred}}

{\ttfamily\noindent\color{warningcolor}{\#\# Warning: fit\_glm: fitted probabilities numerically 0 or 1 occurred}}\end{kframe}
\end{knitrout}
\begin{knitrout}
\definecolor{shadecolor}{rgb}{0.969, 0.969, 0.969}\color{fgcolor}\begin{kframe}
\begin{alltt}
\hlkwd{summary}\hlstd{(logistic_regression_result)}
\end{alltt}
\begin{verbatim}
## 
## Call:
## fastcpd(data = data, beta = beta, k = function(x) 0, family = "binomial", 
##     epsilon = 1e-05)
## 
## [1] "Residuals:"
##       Min        1Q    Median        3Q       Max 
## -22.20031  -1.01676  -0.00002   1.02329  58.02099 
## 
## Change points:
## [1]  26  73 104 117 144 212 250 267 278
\end{verbatim}
\begin{alltt}
\hlkwd{plot}\hlstd{(logistic_regression_result)}
\end{alltt}
\end{kframe}
\includegraphics[width=\maxwidth]{figure/logistic_regression_plot-1} 
\begin{kframe}\begin{alltt}
\hlcom{# n <- 500}
\hlcom{# p <- 8}
\hlcom{# x <- cbind(1, mvtnorm::rmvnorm(n, rep(0, p - 1), diag(p - 1)))}
\hlcom{# xb <- c(}
\hlcom{#   x[1:225, ] %*% c(-1, rnorm(p - 1, 0.5, 0.1)),}
\hlcom{#   x[226:500, ] %*% c(1, rnorm(p - 1, 0.5, 0.1))}
\hlcom{# )}
\hlcom{# y <- rbinom(n, size = 1, prob = 1 / (1 + exp(-xb)))}
\hlcom{# data <- cbind(y, x)}
\hlcom{# logistic_regression_result <- fastcpd(}
\hlcom{#   data = data,}
\hlcom{#   beta = (p + 1) * log(n) / 2,}
\hlcom{#   family = "binomial"}
\hlcom{# )}
\end{alltt}
\end{kframe}
\end{knitrout}

\newgeometry{margin=1cm}
\begin{landscape}

  \begin{table}[t!]
    \centering
    \begin{tabular}{@{}cccccccccc@{}}
    \toprule
                           & \multicolumn{3}{c}{p=1}                                            & \multicolumn{3}{c}{p=3}                                            & \multicolumn{3}{c}{p=5}                                            \\
                           & n=300                & n=600                & n=1500               & n=300                & n=600                & n=1500               & n=300                & n=600                & n=1500               \\ \midrule
    \multirow{4}{*}{cpc=0} & 0.61, 1.34s & 0.64, 2.96s & 0.61, 8.70s & 0.22, 1.01s & 0.21, 2.16s & 0.30, 6.24s & 0.12, 0.96s & 0.10, 1.93s & 0.11, 5.35s \\
                           & 0.67, 23.88s & 0.66, 101.76s & 0.88, 405.55s & 0.15, 6.41s & 0.11, 14.61s & 0.15, 91.25s & 0.07, 4.94s & 0.07, 10.64s & 0.15, 31.18s \\
                           & 0.77, 45.09s & 0.85, 185.40s & 0.82, 784.56s & 0.09, 6.70s & 0.08, 18.90s & 0.15, 101.19s & 0.07, 7.75s & 0.07, 16.13s & 0.16, 60.53s \\
                           & 0.48, 23.04s & 0.53, 79.22s & 0.65, 238.85s & 0.06, 10.93s & 0.09, 21.36s & 0.20, 78.12s & 0.07, 10.18s & 0.06, 21.70s & 0.27, 64.66s \\ \midrule
    \multirow{4}{*}{cpc=1} & 0.69, 1.47s & 0.78, 3.30s & 0.85, 10.43s & 0.67, 0.91s & 0.65, 2.19s & 0.68, 6.29s & 0.58, 0.94s & 0.60, 1.88s & 0.61, 5.68s \\
                           & 0.64, 22.21s & 0.75, 76.50s & 0.81, 452.43s & 0.63, 4.55s & 0.60, 15.78s & 0.61, 80.27s & 0.57, 4.42s & 0.57, 10.38s & 0.67, 30.22s \\
                           & 0.84, 33.11s & 0.69, 143.66s & 0.68, 681.70s & 0.57, 6.26s & 0.59, 17.69s & 0.64, 128.01s & 0.57, 7.94s & 0.57, 18.31s & 0.61, 81.24s \\
                           & 0.71, 22.97s & 0.70, 73.92s & 0.83, 253.55s & 0.56, 9.30s & 0.59, 22.18s & 0.65, 88.01s & 0.57, 9.86s & 0.56, 24.33s & 0.61, 74.32s \\ \midrule
    \multirow{4}{*}{cpc=3} & 0.76, 1.28s & 0.76, 2.79s & 0.87, 10.19s & 0.85, 0.94s & 0.87, 2.27s & 0.82, 5.94s & 0.83, 0.97s & 0.83, 1.92s & 0.82, 5.16s \\
                           & 0.71, 18.69s & 0.78, 90.92s & 0.75, 405.34s & 0.83, 5.63s & 0.83, 16.40s & 0.86, 86.75s & 0.82, 4.81s & 0.81, 10.35s & 0.82, 37.93s \\
                           & 0.62, 35.17s & 0.76, 99.60s & 0.67, 609.95s & 0.82, 7.69s & 0.82, 25.64s & 0.85, 104.59s & 0.81, 8.07s & 0.81, 15.89s & 0.83, 64.66s \\
                           & 0.58, 28.55s & 0.74, 66.22s & 0.84, 225.26s & 0.82, 12.20s & 0.83, 22.50s & 0.81, 83.54s & 0.82, 11.57s & 0.80, 23.02s & 0.81, 71.49s \\ \midrule
    \multirow{4}{*}{cpc=5} & 0.67, 1.13s & 0.80, 3.01s & 0.78, 9.63s & 0.86, 0.88s & 0.90, 2.02s & 0.90, 6.32s & 0.89, 0.89s & 0.89, 1.79s & 0.88, 5.04s \\
                           & 0.71, 19.48s & 0.72, 59.59s & 0.81, 349.84s & 0.89, 4.97s & 0.88, 12.10s & 0.90, 55.76s & 0.89, 5.49s & 0.88, 9.70s & 0.89, 33.49s \\
                           & 0.83, 38.36s & 0.63, 142.46s & 0.80, 592.42s & 0.89, 8.16s & 0.90, 23.92s & 0.90, 99.00s & 0.90, 8.36s & 0.89, 19.43s & 0.87, 60.18s \\
                           & 0.51, 24.90s & 0.67, 70.35s & 0.83, 221.87s & 0.88, 9.02s & 0.89, 25.20s & 0.87, 68.76s & 0.89, 9.85s & 0.88, 20.84s & 0.80, 49.09s \\ \bottomrule
    \end{tabular}
    \caption{\label{tab:logistic regression comparison} Comparison of the algorithms for the logistic regression model in mean.}
  \end{table}

\end{landscape}
\restoregeometry

% \begin{figure}[t!]
% \centering
% <<logistic regression visualization, echo=FALSE, fig=TRUE, height=5.2, width=7>>=
% benchmarked <- microbenchmark::microbenchmark(
%   "SGD PELT k = 1" = fastcpd(
%     data = data, beta = (p + 1) * log(n) / 2, k = function(x) 0, family = "binomial"
%   ),
%   "SGD PELT k = 3" = fastcpd(
%     data = data, beta = (p + 1) * log(n) / 2, k = function(x) 2, family = "binomial"
%   ),
%   "SGD PELT k = 5" = fastcpd(
%     data = data, beta = (p + 1) * log(n) / 2, k = function(x) 4, family = "binomial"
%   ),
%   "Vanilla PELT" = CP_vanilla_binomial(data[, c(1:p + 1, 1)], (p + 1) * log(n) / 2)$cp,
%   times = 2
% )
% ggplot2::autoplot(benchmarked)
% @
% \caption{\label{fig:logistic regression bench mark} Benchmarking results for
% the logistic regression model.}
% \end{figure}

% \begin{table}[t!]
% \centering
% \begin{tabular}{llp{10cm}}
% \hline
% Algorithm        & Median Time (s)   & Change Points \\ \hline
% SGD PELT $k = 1$ & summary(benchmarked)[1, 5] & fastcpd(data, (p + 1) * log(n) / 2, k = function(x) 0, family = "binomial")@cp_set \\
% SGD PELT $k = 3$ & summary(benchmarked)[2, 5] & fastcpd(data, (p + 1) * log(n) / 2, k = function(x) 2, family = "binomial")@cp_set \\
% SGD PELT $k = 5$ & summary(benchmarked)[3, 5] & fastcpd(data, (p + 1) * log(n) / 2, k = function(x) 4, family = "binomial")@cp_set \\
% Vanilla PELT     & summary(benchmarked)[4, 5] & CP_vanilla_binomial(data[, c(1:p + 1, 1)], (p + 1) * log(n) / 2)$cp \\ \hline
% \end{tabular}
% \caption{\label{tab:logistic regression comparison} Comparison of the
% algorithms for the logistic regression model.}
% \end{table}

\section{Changes in generalized linear models: Poisson regression} \label{sec:poisson}

We now consider the change-point detection problem in the Poisson Regression
setting.
%
\begin{knitrout}
\definecolor{shadecolor}{rgb}{0.969, 0.969, 0.969}\color{fgcolor}\begin{kframe}
\begin{alltt}
\hlstd{data_gen_poisson} \hlkwb{<-} \hlkwa{function}\hlstd{(}\hlkwc{n}\hlstd{,} \hlkwc{p}\hlstd{,} \hlkwc{true.coef}\hlstd{,} \hlkwc{true.cp.loc}\hlstd{,} \hlkwc{Sigma}\hlstd{,} \hlkwc{evar}\hlstd{) \{}
  \hlstd{loc} \hlkwb{<-} \hlkwd{unique}\hlstd{(}\hlkwd{c}\hlstd{(}\hlnum{0}\hlstd{, true.cp.loc, n))}
  \hlkwa{if}\hlstd{(}\hlkwd{dim}\hlstd{(true.coef)[}\hlnum{2}\hlstd{]} \hlopt{!=} \hlkwd{length}\hlstd{(loc)}\hlopt{-}\hlnum{1}\hlstd{)} \hlkwd{stop}\hlstd{(}\hlstr{"true.coef and true.cp.loc do not match"}\hlstd{)}
  \hlstd{x} \hlkwb{<-} \hlstd{mvtnorm}\hlopt{::}\hlkwd{rmvnorm}\hlstd{(n,} \hlkwc{mean} \hlstd{=} \hlkwd{rep}\hlstd{(}\hlnum{0}\hlstd{, p),} \hlkwc{sigma} \hlstd{= Sigma)}
  \hlstd{y} \hlkwb{<-} \hlkwa{NULL}
  \hlkwa{for}\hlstd{(i} \hlkwa{in} \hlnum{1}\hlopt{:}\hlstd{(}\hlkwd{length}\hlstd{(loc)}\hlopt{-}\hlnum{1}\hlstd{))}
  \hlstd{\{}
    \hlstd{Xb} \hlkwb{<-} \hlstd{x[(loc[i]} \hlopt{+} \hlnum{1}\hlstd{)}\hlopt{:}\hlstd{loc[i} \hlopt{+} \hlnum{1}\hlstd{], ,}\hlkwc{drop}\hlstd{=}\hlnum{FALSE}\hlstd{]}\hlopt\hlstd{true.coef[,i,}\hlkwc{drop}\hlstd{=}\hlnum{FALSE}\hlstd{]}
    \hlstd{y} \hlkwb{<-} \hlkwd{c}\hlstd{(y,} \hlkwd{rpois}\hlstd{(loc[i} \hlopt{+} \hlnum{1}\hlstd{]} \hlopt{-} \hlstd{loc[i],} \hlkwd{exp}\hlstd{(Xb)))}
  \hlstd{\}}
  \hlstd{data} \hlkwb{<-} \hlkwd{cbind}\hlstd{(y, x)}
  \hlstd{true_cluster} \hlkwb{<-} \hlkwd{rep}\hlstd{(}\hlnum{1}\hlopt{:}\hlstd{(}\hlkwd{length}\hlstd{(loc)}\hlopt{-}\hlnum{1}\hlstd{),} \hlkwd{diff}\hlstd{(loc))}
  \hlstd{result} \hlkwb{<-} \hlkwd{list}\hlstd{(data, true_cluster)}
  \hlkwd{return}\hlstd{(result)}
\hlstd{\}}

\hlstd{cost_glm_poisson} \hlkwb{<-} \hlkwa{function}\hlstd{(}\hlkwc{data}\hlstd{,} \hlkwc{family}\hlstd{=}\hlstr{"poisson"}\hlstd{)}
\hlstd{\{}
\hlstd{data} \hlkwb{<-} \hlkwd{as.matrix}\hlstd{(data)}
\hlstd{p} \hlkwb{<-} \hlkwd{dim}\hlstd{(data)[}\hlnum{2}\hlstd{]} \hlopt{-} \hlnum{1}
\hlstd{out} \hlkwb{<-} \hlkwd{fastglm}\hlstd{(}\hlkwd{as.matrix}\hlstd{(data[,}\hlnum{1}\hlopt{:}\hlstd{p]), data[,p}\hlopt{+}\hlnum{1}\hlstd{],} \hlkwc{family}\hlstd{=family)}
\hlkwd{return}\hlstd{(out}\hlopt{$}\hlstd{deviance}\hlopt{/}\hlnum{2}\hlstd{)}
\hlstd{\}}

\hlstd{CP_vanilla_poisson} \hlkwb{<-} \hlkwa{function}\hlstd{(}\hlkwc{data}\hlstd{,} \hlkwc{beta}\hlstd{,} \hlkwc{cost}\hlstd{=cost_glm_poisson,} \hlkwc{trim} \hlstd{=} \hlnum{0.025}\hlstd{)}
\hlstd{\{}
\hlstd{n} \hlkwb{<-} \hlkwd{dim}\hlstd{(data)[}\hlnum{1}\hlstd{]}
\hlstd{p} \hlkwb{<-} \hlkwd{dim}\hlstd{(data)[}\hlnum{2}\hlstd{]} \hlopt{-} \hlnum{1}
\hlstd{Fobj} \hlkwb{<-} \hlkwd{c}\hlstd{(}\hlopt{-}\hlstd{beta,} \hlnum{0}\hlstd{)}
\hlstd{cp_set} \hlkwb{<-} \hlkwd{list}\hlstd{(}\hlkwa{NULL}\hlstd{,}\hlnum{0}\hlstd{)}
\hlstd{set} \hlkwb{<-} \hlkwd{c}\hlstd{(}\hlnum{0}\hlstd{,}\hlnum{1}\hlstd{)}
 \hlkwa{for}\hlstd{(t} \hlkwa{in} \hlnum{2}\hlopt{:}\hlstd{n)}
 \hlstd{\{}
 \hlstd{m} \hlkwb{<-} \hlkwd{length}\hlstd{(set)}
 \hlstd{cval} \hlkwb{<-} \hlkwd{rep}\hlstd{(}\hlnum{NA}\hlstd{, m)}
  \hlkwa{for}\hlstd{(i} \hlkwa{in} \hlnum{1}\hlopt{:}\hlstd{m)}
  \hlstd{\{}
  \hlstd{k} \hlkwb{<-} \hlstd{set[i]} \hlopt{+} \hlnum{1}
  \hlkwa{if}\hlstd{(t}\hlopt{-}\hlstd{k}\hlopt{>=}\hlstd{p}\hlopt{-}\hlnum{1}\hlstd{) cval[i]} \hlkwb{<-} \hlkwd{suppressWarnings}\hlstd{(}\hlkwd{cost}\hlstd{(data[k}\hlopt{:}\hlstd{t,,} \hlkwc{drop} \hlstd{=} \hlnum{FALSE}\hlstd{]))} \hlkwa{else} \hlstd{cval[i]} \hlkwb{<-} \hlnum{0}
  \hlstd{\}}
 \hlstd{obj} \hlkwb{<-} \hlstd{cval} \hlopt{+} \hlstd{Fobj[set}\hlopt{+}\hlnum{1}\hlstd{]} \hlopt{+} \hlstd{beta}
 \hlstd{min_val} \hlkwb{<-} \hlkwd{min}\hlstd{(obj)}
 \hlstd{ind} \hlkwb{<-} \hlkwd{which}\hlstd{(obj}\hlopt{==}\hlstd{min_val)[}\hlnum{1}\hlstd{]}
 \hlstd{cp_set_add} \hlkwb{<-} \hlkwd{c}\hlstd{(cp_set[[set[ind]}\hlopt{+}\hlnum{1}\hlstd{]], set[ind])}
 \hlstd{cp_set} \hlkwb{<-} \hlkwd{append}\hlstd{(cp_set,}\hlkwd{list}\hlstd{(cp_set_add))}
 \hlstd{ind2} \hlkwb{<-} \hlstd{(cval} \hlopt{+} \hlstd{Fobj[set}\hlopt{+}\hlnum{1}\hlstd{])} \hlopt{<=} \hlstd{min_val}
 \hlstd{set} \hlkwb{<-} \hlkwd{c}\hlstd{(set[ind2], t)}
 \hlstd{Fobj} \hlkwb{<-} \hlkwd{c}\hlstd{(Fobj, min_val)}
 \hlstd{\}}
\hlstd{cp} \hlkwb{<-} \hlstd{cp_set[[n}\hlopt{+}\hlnum{1}\hlstd{]]}
\hlstd{nLL} \hlkwb{<-} \hlnum{0}

\hlstd{cp} \hlkwb{<-} \hlstd{cp[(cp} \hlopt{>=} \hlstd{trim} \hlopt{*} \hlstd{n)} \hlopt{&} \hlstd{(cp} \hlopt{<=} \hlstd{(}\hlnum{1} \hlopt{-} \hlstd{trim)} \hlopt{*} \hlstd{n)]}
\hlstd{cp} \hlkwb{<-} \hlkwd{sort}\hlstd{(}\hlkwd{unique}\hlstd{(}\hlkwd{c}\hlstd{(}\hlnum{0}\hlstd{, cp)))}

\hlstd{segment_indices} \hlkwb{<-} \hlkwd{which}\hlstd{((}\hlkwd{diff}\hlstd{(cp)} \hlopt{<} \hlstd{trim} \hlopt{*} \hlstd{n)} \hlopt{==} \hlnum{TRUE}\hlstd{)}
\hlkwa{if} \hlstd{(}\hlkwd{length}\hlstd{(segment_indices)} \hlopt{>} \hlnum{0}\hlstd{) \{}
  \hlstd{cp} \hlkwb{<-} \hlkwd{floor}\hlstd{(}
    \hlstd{(cp[}\hlopt{-}\hlstd{(segment_indices} \hlopt{+} \hlnum{1}\hlstd{)]} \hlopt{+} \hlstd{cp[}\hlopt{-}\hlstd{segment_indices])} \hlopt{/} \hlnum{2}
  \hlstd{)}
\hlstd{\}}
\hlstd{cp_loc} \hlkwb{<-} \hlkwd{unique}\hlstd{(}\hlkwd{c}\hlstd{(}\hlnum{0}\hlstd{,cp,n))}
\hlkwa{for}\hlstd{(i} \hlkwa{in} \hlnum{1}\hlopt{:}\hlstd{(}\hlkwd{length}\hlstd{(cp_loc)}\hlopt{-}\hlnum{1}\hlstd{))}
 \hlstd{\{}
 \hlstd{seg} \hlkwb{<-} \hlstd{(cp_loc[i]}\hlopt{+}\hlnum{1}\hlstd{)}\hlopt{:}\hlstd{cp_loc[i}\hlopt{+}\hlnum{1}\hlstd{]}
 \hlstd{data_seg} \hlkwb{<-} \hlstd{data[seg,]}
 \hlstd{out} \hlkwb{<-} \hlkwd{fastglm}\hlstd{(}\hlkwd{as.matrix}\hlstd{(data_seg[,} \hlnum{1}\hlopt{:}\hlstd{p]), data_seg[, p}\hlopt{+}\hlnum{1}\hlstd{],} \hlkwc{family}\hlstd{=}\hlstr{"poisson"}\hlstd{)}
 \hlstd{nLL} \hlkwb{<-} \hlstd{out}\hlopt{$}\hlstd{deviance}\hlopt{/}\hlnum{2} \hlopt{+} \hlstd{nLL}
 \hlstd{\}}

\hlstd{output} \hlkwb{<-} \hlkwd{list}\hlstd{(cp, nLL)}
\hlkwd{names}\hlstd{(output)} \hlkwb{<-} \hlkwd{c}\hlstd{(}\hlstr{"cp"}\hlstd{,} \hlstr{"nLL"}\hlstd{)}
\hlkwd{return}\hlstd{(output)}
\hlstd{\}}
\hlstd{experiment_setup} \hlkwb{<-} \hlstd{purrr}\hlopt{::}\hlkwd{cross_df}\hlstd{(}
  \hlkwd{list}\hlstd{(}
    \hlkwc{n} \hlstd{=} \hlkwd{c}\hlstd{(}\hlnum{300}\hlstd{,} \hlnum{600}\hlstd{,} \hlnum{1500}\hlstd{),}
    \hlkwc{p} \hlstd{=} \hlkwd{c}\hlstd{(}\hlnum{1}\hlstd{,} \hlnum{3}\hlstd{,} \hlnum{5}\hlstd{),}
    \hlkwc{change_points_count} \hlstd{=} \hlkwd{c}\hlstd{(}\hlnum{0}\hlstd{,} \hlnum{1}\hlstd{,} \hlnum{3}\hlstd{,} \hlnum{5}\hlstd{),}
    \hlkwc{sgd_k} \hlstd{=} \hlkwd{c}\hlstd{(}\hlnum{1}\hlstd{,} \hlnum{3}\hlstd{,} \hlnum{5}\hlstd{,} \hlnum{7}\hlstd{)}
  \hlstd{)}
\hlstd{)}
\hlstd{experiment_num} \hlkwb{<-} \hlnum{10}

\hlstd{rand_gd} \hlkwb{<-} \hlstd{time_gd} \hlkwb{<-} \hlkwd{rep}\hlstd{(}\hlkwd{list}\hlstd{(}\hlkwd{rep}\hlstd{(}\hlnum{NA}\hlstd{, experiment_num)),} \hlkwd{nrow}\hlstd{(experiment_setup))}
\hlstd{rng} \hlkwb{<-} \hlkwd{RNGseq}\hlstd{(experiment_num} \hlopt{*} \hlkwd{nrow}\hlstd{(experiment_setup),} \hlnum{3}\hlstd{)}

\hlstd{cl} \hlkwb{<-} \hlstd{parallel}\hlopt{::}\hlkwd{makeCluster}\hlstd{(parallel}\hlopt{::}\hlkwd{detectCores}\hlstd{(),} \hlkwc{outfile} \hlstd{=} \hlkwd{paste0}\hlstd{(}\hlstr{"log/poisson.log"}\hlstd{))}
\hlstd{doParallel}\hlopt{::}\hlkwd{registerDoParallel}\hlstd{(cl)}
\hlstd{experiment_result} \hlkwb{<-} \hlstd{foreach}\hlopt{::}\hlkwd{foreach}\hlstd{(}\hlkwc{experiment_setup_index} \hlstd{=} \hlkwd{seq_len}\hlstd{(}\hlkwd{nrow}\hlstd{(experiment_setup)),} \hlkwc{.packages} \hlstd{=} \hlkwd{c}\hlstd{(}\hlstr{"fastglm"}\hlstd{,} \hlstr{"fastcpd"}\hlstd{,} \hlstr{"fossil"}\hlstd{))} \hlopt
  \hlstd{foreach}\hlopt{::}\hlkwd{foreach}\hlstd{(}\hlkwc{experiment_index} \hlstd{=} \hlkwd{seq_len}\hlstd{(}\hlnum{5}\hlstd{),} \hlkwc{rrng} \hlstd{= rng[(experiment_setup_index} \hlopt{-} \hlnum{1}\hlstd{)} \hlopt{*} \hlstd{experiment_num} \hlopt{+} \hlkwd{seq_len}\hlstd{(}\hlnum{5}\hlstd{)],} \hlkwc{.combine} \hlstd{=} \hlstr{"rbind"}\hlstd{)} \hlopt \hlstd{\{}
    \hlstd{rngtools}\hlopt{::}\hlkwd{setRNG}\hlstd{(rrng)}
    \hlstd{experiment_setup_row} \hlkwb{<-} \hlstd{experiment_setup[experiment_setup_index, ]}
    \hlstd{n} \hlkwb{<-} \hlstd{experiment_setup_row}\hlopt{$}\hlstd{n}
    \hlstd{p} \hlkwb{<-} \hlstd{experiment_setup_row}\hlopt{$}\hlstd{p}
    \hlstd{change_points_count} \hlkwb{<-} \hlstd{experiment_setup_row}\hlopt{$}\hlstd{change_points_count}
    \hlstd{sgd_k} \hlkwb{<-} \hlstd{experiment_setup_row}\hlopt{$}\hlstd{sgd_k}

    \hlkwa{if} \hlstd{(p} \hlopt{==} \hlnum{1}\hlstd{) \{}
      \hlstd{theta_0} \hlkwb{<-} \hlnum{1.2}
    \hlstd{\}} \hlkwa{else} \hlstd{\{}
      \hlstd{theta_0} \hlkwb{<-} \hlkwd{c}\hlstd{(}\hlnum{1}\hlstd{,} \hlnum{1.2}\hlstd{,} \hlopt{-}\hlnum{1}\hlstd{,} \hlnum{0.5}\hlstd{,} \hlopt{-}\hlnum{2}\hlstd{)[}\hlkwd{seq_len}\hlstd{(p)]}
    \hlstd{\}}

    \hlstd{true.coef} \hlkwb{<-} \hlkwd{matrix}\hlstd{(}\hlkwd{rep}\hlstd{(theta_0, change_points_count} \hlopt{+} \hlnum{1}\hlstd{),} \hlkwc{nrow} \hlstd{= p,} \hlkwc{ncol} \hlstd{= change_points_count} \hlopt{+} \hlnum{1}\hlstd{)}

    \hlstd{theta_norm} \hlkwb{<-} \hlkwd{c}\hlstd{(}\hlnum{0.36}\hlstd{,} \hlnum{NA}\hlstd{,} \hlnum{0.81}\hlstd{,} \hlnum{NA}\hlstd{,} \hlnum{1.96}\hlstd{)[p]}
    \hlstd{delta} \hlkwb{<-} \hlkwd{rnorm}\hlstd{(p)}
    \hlstd{sigma_} \hlkwb{<-} \hlnum{0.9}\hlopt{**}\hlkwd{abs}\hlstd{(}\hlkwd{row}\hlstd{(}\hlkwd{diag}\hlstd{(p))} \hlopt{-} \hlkwd{col}\hlstd{(}\hlkwd{diag}\hlstd{(p)))}
    \hlstd{delta_coef} \hlkwb{<-} \hlkwd{sqrt}\hlstd{(theta_norm} \hlopt{/} \hlkwd{t}\hlstd{(delta)} \hlopt \hlstd{sigma_} \hlopt \hlstd{delta)}

    \hlkwa{if} \hlstd{(change_points_count} \hlopt{==} \hlnum{0}\hlstd{) \{}
      \hlstd{true.cp.loc} \hlkwb{<-} \hlkwa{NULL}
    \hlstd{\}} \hlkwa{else if} \hlstd{(change_points_count} \hlopt{==} \hlnum{1}\hlstd{) \{}
      \hlstd{true.cp.loc} \hlkwb{<-} \hlstd{n} \hlopt{/} \hlnum{2} \hlopt{*} \hlkwd{seq_len}\hlstd{(change_points_count)}
      \hlstd{true.coef[,} \hlnum{2}\hlstd{]} \hlkwb{<-} \hlstd{theta_0} \hlopt{+} \hlkwd{c}\hlstd{(delta_coef)} \hlopt{*} \hlstd{delta}
    \hlstd{\}} \hlkwa{else if} \hlstd{(change_points_count} \hlopt{==} \hlnum{3}\hlstd{) \{}
      \hlstd{true.cp.loc} \hlkwb{<-} \hlstd{n} \hlopt{/} \hlnum{4} \hlopt{*} \hlkwd{seq_len}\hlstd{(change_points_count)}
      \hlstd{true.coef[,} \hlnum{2}\hlstd{]} \hlkwb{<-} \hlstd{theta_0} \hlopt{+} \hlkwd{c}\hlstd{(delta_coef)} \hlopt{*} \hlstd{delta}
      \hlstd{true.coef[,} \hlnum{4}\hlstd{]} \hlkwb{<-} \hlstd{theta_0} \hlopt{-} \hlkwd{c}\hlstd{(delta_coef)} \hlopt{*} \hlstd{delta}
    \hlstd{\}} \hlkwa{else if} \hlstd{(change_points_count} \hlopt{==} \hlnum{5}\hlstd{) \{}
      \hlstd{true.cp.loc} \hlkwb{<-} \hlstd{n} \hlopt{/} \hlnum{6} \hlopt{*} \hlkwd{seq_len}\hlstd{(change_points_count)}
      \hlstd{true.coef[,} \hlnum{2}\hlstd{]} \hlkwb{<-} \hlstd{theta_0} \hlopt{+} \hlkwd{c}\hlstd{(delta_coef)} \hlopt{*} \hlstd{delta}
      \hlstd{true.coef[,} \hlnum{4}\hlstd{]} \hlkwb{<-} \hlstd{theta_0} \hlopt{-} \hlkwd{c}\hlstd{(delta_coef)} \hlopt{*} \hlstd{delta}
      \hlstd{true.coef[,} \hlnum{6}\hlstd{]} \hlkwb{<-} \hlstd{theta_0} \hlopt{+} \hlkwd{c}\hlstd{(delta_coef)} \hlopt{*} \hlstd{delta}
    \hlstd{\}}

    \hlstd{Sigma} \hlkwb{<-} \hlkwd{diag}\hlstd{(}\hlnum{1}\hlstd{, p)}
    \hlstd{evar} \hlkwb{<-} \hlnum{0.5}
    \hlstd{out} \hlkwb{<-} \hlkwd{data_gen_poisson}\hlstd{(n, p, true.coef, true.cp.loc, Sigma, evar)}
    \hlstd{data} \hlkwb{<-} \hlstd{out[[}\hlnum{1}\hlstd{]]}
    \hlstd{g_tr} \hlkwb{<-} \hlstd{out[[}\hlnum{2}\hlstd{]]}
    \hlstd{beta} \hlkwb{<-} \hlkwd{log}\hlstd{(n)}\hlopt{/}\hlnum{2}

    \hlkwa{if} \hlstd{(}\hlkwd{file.exists}\hlstd{(}\hlkwd{paste0}\hlstd{(}\hlstr{"cache/poisson"}\hlstd{, experiment_setup_index,} \hlstr{"_"}\hlstd{, experiment_index,} \hlstr{".rds"}\hlstd{))) \{}
      \hlstd{rds_file} \hlkwb{<-} \hlkwd{readRDS}\hlstd{(}\hlkwd{paste0}\hlstd{(}\hlstr{"cache/poisson"}\hlstd{, experiment_setup_index,} \hlstr{"_"}\hlstd{, experiment_index,} \hlstr{".rds"}\hlstd{))}
      \hlstd{cp_set} \hlkwb{<-} \hlstd{rds_file}\hlopt{$}\hlstd{cp_set}
      \hlstd{time_used} \hlkwb{<-} \hlstd{rds_file}\hlopt{$}\hlstd{time_used}
    \hlstd{\}} \hlkwa{else} \hlstd{\{}
      \hlstd{start} \hlkwb{<-} \hlkwd{proc.time}\hlstd{()}
      \hlkwa{if} \hlstd{(sgd_k} \hlopt{==} \hlnum{7}\hlstd{) \{}
        \hlstd{cp_set} \hlkwb{<-} \hlkwd{CP_vanilla_poisson}\hlstd{(data[,} \hlkwd{c}\hlstd{(}\hlnum{1}\hlopt{:}\hlstd{p} \hlopt{+} \hlnum{1}\hlstd{,} \hlnum{1}\hlstd{)], beta)}\hlopt{$}\hlstd{cp}
      \hlstd{\}} \hlkwa{else} \hlstd{\{}
        \hlstd{poisson_regression_result} \hlkwb{<-} \hlkwd{fastcpd}\hlstd{(}
          \hlkwc{data} \hlstd{= data,}
          \hlkwc{beta} \hlstd{= beta,}
          \hlkwc{family} \hlstd{=} \hlstr{"poisson"}\hlstd{,}
          \hlkwc{k} \hlstd{=} \hlkwa{function}\hlstd{(}\hlkwc{x}\hlstd{) sgd_k} \hlopt{-} \hlnum{1}\hlstd{,}
          \hlkwc{epsilon} \hlstd{=} \hlnum{1e-5}
        \hlstd{)}
        \hlstd{cp_set} \hlkwb{<-} \hlstd{poisson_regression_result}\hlopt{@}\hlkwc{cp_set}
      \hlstd{\}}
      \hlstd{time_used} \hlkwb{<-} \hlkwd{unname}\hlstd{((}\hlkwd{proc.time}\hlstd{()} \hlopt{-} \hlstd{start)[}\hlnum{3}\hlstd{])}
      \hlkwd{saveRDS}\hlstd{(}\hlkwd{list}\hlstd{(}\hlkwc{data} \hlstd{= data,} \hlkwc{cp_set} \hlstd{= cp_set,} \hlkwc{time_used} \hlstd{= time_used),} \hlkwd{paste0}\hlstd{(}\hlstr{"cache/poisson"}\hlstd{, experiment_setup_index,} \hlstr{"_"}\hlstd{, experiment_index,} \hlstr{".rds"}\hlstd{))}
    \hlstd{\}}
    \hlstd{cp_gd} \hlkwb{<-} \hlstd{cp_set[}\hlopt{!}\hlstd{(cp_set}\hlopt{==}\hlnum{0}\hlstd{)]}
    \hlstd{K_est} \hlkwb{<-} \hlkwd{length}\hlstd{(cp_gd)} \hlopt{+} \hlnum{1}
    \hlstd{cp_un} \hlkwb{<-} \hlkwd{unique}\hlstd{(}\hlkwd{c}\hlstd{(}\hlnum{0}\hlstd{, cp_gd, n))}
    \hlstd{g_est} \hlkwb{<-} \hlkwd{rep}\hlstd{(}\hlnum{1}\hlopt{:}\hlstd{K_est,} \hlkwd{diff}\hlstd{(cp_un))}
    \hlkwd{c}\hlstd{(fossil}\hlopt{::}\hlkwd{rand.index}\hlstd{(g_tr, g_est), time_used)}
  \hlstd{\}}
\hlstd{parallel}\hlopt{::}\hlkwd{stopCluster}\hlstd{(cl)}

\hlstd{experiment_setup_index} \hlkwb{<-} \hlnum{25}
\hlstd{experiment_setup_row} \hlkwb{<-} \hlstd{experiment_setup[experiment_setup_index, ]}
\hlstd{n} \hlkwb{<-} \hlstd{experiment_setup_row}\hlopt{$}\hlstd{n}
\hlstd{p} \hlkwb{<-} \hlstd{experiment_setup_row}\hlopt{$}\hlstd{p}
\hlstd{change_points_count} \hlkwb{<-} \hlstd{experiment_setup_row}\hlopt{$}\hlstd{change_points_count}
\hlstd{sgd_k} \hlkwb{<-} \hlstd{experiment_setup_row}\hlopt{$}\hlstd{sgd_k}

\hlkwa{if} \hlstd{(p} \hlopt{==} \hlnum{1}\hlstd{) \{}
  \hlstd{theta_0} \hlkwb{<-} \hlnum{1.2}
\hlstd{\}} \hlkwa{else} \hlstd{\{}
  \hlstd{theta_0} \hlkwb{<-} \hlkwd{c}\hlstd{(}\hlnum{1}\hlstd{,} \hlnum{1.2}\hlstd{,} \hlopt{-}\hlnum{1}\hlstd{,} \hlnum{0.5}\hlstd{,} \hlopt{-}\hlnum{2}\hlstd{)[}\hlkwd{seq_len}\hlstd{(p)]}
\hlstd{\}}

\hlstd{true.coef} \hlkwb{<-} \hlkwd{matrix}\hlstd{(}\hlkwd{rep}\hlstd{(theta_0, change_points_count} \hlopt{+} \hlnum{1}\hlstd{),} \hlkwc{nrow} \hlstd{= p,} \hlkwc{ncol} \hlstd{= change_points_count} \hlopt{+} \hlnum{1}\hlstd{)}

\hlstd{theta_norm} \hlkwb{<-} \hlkwd{c}\hlstd{(}\hlnum{0.36}\hlstd{,} \hlnum{NA}\hlstd{,} \hlnum{0.81}\hlstd{,} \hlnum{NA}\hlstd{,} \hlnum{1.96}\hlstd{)[p]}
\hlstd{delta} \hlkwb{<-} \hlkwd{rnorm}\hlstd{(p)}
\hlstd{sigma_} \hlkwb{<-} \hlnum{0.9}\hlopt{**}\hlkwd{abs}\hlstd{(}\hlkwd{row}\hlstd{(}\hlkwd{diag}\hlstd{(p))} \hlopt{-} \hlkwd{col}\hlstd{(}\hlkwd{diag}\hlstd{(p)))}
\hlstd{delta_coef} \hlkwb{<-} \hlkwd{sqrt}\hlstd{(theta_norm} \hlopt{/} \hlkwd{t}\hlstd{(delta)} \hlopt \hlstd{sigma_} \hlopt \hlstd{delta)}

\hlkwa{if} \hlstd{(change_points_count} \hlopt{==} \hlnum{0}\hlstd{) \{}
  \hlstd{true.cp.loc} \hlkwb{<-} \hlkwa{NULL}
\hlstd{\}} \hlkwa{else if} \hlstd{(change_points_count} \hlopt{==} \hlnum{1}\hlstd{) \{}
  \hlstd{true.cp.loc} \hlkwb{<-} \hlstd{n} \hlopt{/} \hlnum{2} \hlopt{*} \hlkwd{seq_len}\hlstd{(change_points_count)}
  \hlstd{true.coef[,} \hlnum{2}\hlstd{]} \hlkwb{<-} \hlstd{theta_0} \hlopt{+} \hlkwd{c}\hlstd{(delta_coef)} \hlopt{*} \hlstd{delta}
\hlstd{\}} \hlkwa{else if} \hlstd{(change_points_count} \hlopt{==} \hlnum{3}\hlstd{) \{}
  \hlstd{true.cp.loc} \hlkwb{<-} \hlstd{n} \hlopt{/} \hlnum{4} \hlopt{*} \hlkwd{seq_len}\hlstd{(change_points_count)}
  \hlstd{true.coef[,} \hlnum{2}\hlstd{]} \hlkwb{<-} \hlstd{theta_0} \hlopt{+} \hlkwd{c}\hlstd{(delta_coef)} \hlopt{*} \hlstd{delta}
  \hlstd{true.coef[,} \hlnum{4}\hlstd{]} \hlkwb{<-} \hlstd{theta_0} \hlopt{-} \hlkwd{c}\hlstd{(delta_coef)} \hlopt{*} \hlstd{delta}
\hlstd{\}} \hlkwa{else if} \hlstd{(change_points_count} \hlopt{==} \hlnum{5}\hlstd{) \{}
  \hlstd{true.cp.loc} \hlkwb{<-} \hlstd{n} \hlopt{/} \hlnum{6} \hlopt{*} \hlkwd{seq_len}\hlstd{(change_points_count)}
  \hlstd{true.coef[,} \hlnum{2}\hlstd{]} \hlkwb{<-} \hlstd{theta_0} \hlopt{+} \hlkwd{c}\hlstd{(delta_coef)} \hlopt{*} \hlstd{delta}
  \hlstd{true.coef[,} \hlnum{4}\hlstd{]} \hlkwb{<-} \hlstd{theta_0} \hlopt{-} \hlkwd{c}\hlstd{(delta_coef)} \hlopt{*} \hlstd{delta}
  \hlstd{true.coef[,} \hlnum{6}\hlstd{]} \hlkwb{<-} \hlstd{theta_0} \hlopt{+} \hlkwd{c}\hlstd{(delta_coef)} \hlopt{*} \hlstd{delta}
\hlstd{\}}

\hlstd{Sigma} \hlkwb{<-} \hlkwd{diag}\hlstd{(}\hlnum{1}\hlstd{, p)}
\hlstd{evar} \hlkwb{<-} \hlnum{0.5}
\hlstd{out} \hlkwb{<-} \hlkwd{data_gen_poisson}\hlstd{(n, p, true.coef, true.cp.loc, Sigma, evar)}
\hlstd{data} \hlkwb{<-} \hlstd{out[[}\hlnum{1}\hlstd{]]}
\hlstd{g_tr} \hlkwb{<-} \hlstd{out[[}\hlnum{2}\hlstd{]]}
\hlstd{beta} \hlkwb{<-} \hlkwd{log}\hlstd{(n)}\hlopt{/}\hlnum{2}
\hlstd{poisson_regression_result} \hlkwb{<-} \hlkwd{fastcpd}\hlstd{(}
  \hlkwc{data} \hlstd{= data,}
  \hlkwc{beta} \hlstd{= beta,}
  \hlkwc{family} \hlstd{=} \hlstr{"poisson"}\hlstd{,}
  \hlkwc{k} \hlstd{=} \hlkwa{function}\hlstd{(}\hlkwc{x}\hlstd{)} \hlnum{0}\hlstd{,}
  \hlkwc{epsilon} \hlstd{=} \hlnum{1e-4}
\hlstd{)}
\end{alltt}
\end{kframe}
\end{knitrout}
%

\begin{knitrout}
\definecolor{shadecolor}{rgb}{0.969, 0.969, 0.969}\color{fgcolor}\begin{kframe}
\begin{alltt}
\hlkwd{summary}\hlstd{(poisson_regression_result)}
\end{alltt}
\begin{verbatim}
## 
## Call:
## fastcpd(data = data, beta = beta, k = function(x) 0, family = "poisson", 
##     epsilon = 1e-04)
## 
## [1] "Residuals:"
##       Min        1Q    Median        3Q       Max 
## -1.000000 -1.000000 -0.577591  0.372314 12.995381 
## 
## Change points:
## [1] 150
\end{verbatim}
\begin{alltt}
\hlkwd{plot}\hlstd{(poisson_regression_result)}
\end{alltt}
\end{kframe}
\includegraphics[width=\maxwidth]{figure/poisson_regression_plot-1} 
\end{knitrout}

\newgeometry{margin=1cm}
\begin{landscape}

  \begin{table}[t!]
    \centering
    \begin{tabular}{@{}cccccccccc@{}}
    \toprule
                           & \multicolumn{3}{c}{p=1}                                            & \multicolumn{3}{c}{p=3}                                            & \multicolumn{3}{c}{p=5}                                            \\
                           & n=300                & n=600                & n=1500               & n=300                & n=600                & n=1500               & n=300                & n=600                & n=1500               \\ \midrule
    \multirow{4}{*}{cpc=0} & 0.57, 4.11s & 0.69, 5.24s & 0.65, 9.85s & 0.14, 3.62s & 0.12, 4.08s & 0.12, 6.50s & 0.46, 7.12s & 0.50, 10.33s & 0.50, 14.94s \\
                           & 0.71, 24.17s & 0.64, 62.72s & 0.64, 413.16s & 0.21, 2.49s & 0.17, 10.75s & 0.15, 45.75s & 0.47, 8.53s & 0.50, 15.08s & 0.50, 33.78s \\
                           & 0.58, 41.26s & 0.41, 145.26s & 0.51, 739.57s & 0.20, 4.70s & 0.16, 16.23s & 0.18, 61.29s & 0.44, 10.64s & 0.50, 15.78s & 0.50, 36.28s \\
                           & 0.74, 38.22s & 0.64, 102.88s & 0.80, 541.88s & 0.15, 20.08s & 0.17, 66.15s & 0.12, 212.00s & 0.08, 16.57s & 0.09, 48.26s & 0.09, 159.42s \\ \midrule
    \multirow{4}{*}{cpc=1} & 0.88, 3.87s & 0.86, 5.36s & 0.80, 7.44s & 0.64, 0.62s & 0.62, 1.30s & 0.63, 3.76s & 0.98, 6.45s & 0.93, 8.36s & 1.00, 14.05s \\
                           & 0.83, 18.51s & 0.92, 71.90s & 0.95, 442.40s & 0.63, 3.56s & 0.64, 10.45s & 0.63, 42.62s & 0.90, 9.38s & 0.97, 17.97s & 1.00, 32.82s \\
                           & 0.84, 26.55s & 0.86, 117.91s & 0.84, 739.52s & 0.63, 6.38s & 0.62, 18.79s & 0.67, 67.00s & 0.93, 7.69s & 1.00, 14.66s & 1.00, 34.34s \\
                           & 0.90, 26.55s & 0.92, 88.83s & 0.90, 440.15s & 0.69, 17.74s & 0.67, 58.88s & 0.62, 207.08s & 0.58, 19.86s & 0.59, 56.14s & 0.60, 156.02s \\ \midrule
    \multirow{4}{*}{cpc=3} & 0.92, 0.94s & 0.94, 2.19s & 0.94, 6.29s & 0.83, 0.59s & 0.84, 1.25s & 0.82, 3.73s & 0.76, 7.97s & 0.75, 9.53s & 0.76, 13.27s \\
                           & 0.88, 12.19s & 0.92, 43.29s & 0.94, 219.21s & 0.83, 2.69s & 0.83, 5.79s & 0.85, 38.09s & 0.74, 10.30s & 0.75, 14.61s & 0.75, 33.68s \\
                           & 0.91, 29.64s & 0.95, 110.69s & 0.95, 575.92s & 0.82, 5.40s & 0.81, 12.36s & 0.83, 51.83s & 0.75, 7.57s & 0.75, 16.40s & 0.75, 11.59s \\
                           & 0.96, 26.71s & 0.94, 77.56s & 0.99, 309.25s & 0.87, 15.29s & 0.85, 50.04s & 0.88, 214.72s & 0.81, 16.48s & 0.84, 50.63s & 0.82, 150.00s \\ \midrule
    \multirow{4}{*}{cpc=5} & 0.92, 1.18s & 0.96, 2.33s & 0.97, 5.91s & 0.87, 0.60s & 0.88, 1.34s & 0.90, 5.44s & 0.67, 6.54s & 0.67, 9.73s & 0.67, 11.56s \\
                           & 0.89, 10.81s & 0.95, 45.44s & 0.96, 204.24s & 0.88, 2.68s & 0.87, 10.02s & 0.86, 36.15s & 0.69, 12.01s & 0.68, 15.26s & 0.67, 30.63s \\
                           & 0.94, 21.44s & 0.96, 76.84s & 0.97, 403.26s & 0.89, 5.56s & 0.87, 11.39s & 0.87, 46.96s & 0.68, 8.50s & 0.67, 13.21s & 0.67, 11.73s \\
                           & 0.96, 22.01s & 0.97, 61.44s & 0.98, 220.29s & 0.91, 16.81s & 0.92, 44.63s & 0.93, 153.21s & 0.89, 14.07s & 0.90, 37.67s & 0.90, 107.41s \\ \bottomrule
    \end{tabular}
    \caption{\label{tab:poisson regression comparison} Comparison of the algorithms for the poisson regression model in mean.}
  \end{table}

\end{landscape}
\restoregeometry

\section{Changes in penalized linear models} \label{sec:penalized linear model}

We now consider the change-point detection problem in the penalized linear
models.
%

%

\begin{knitrout}
\definecolor{shadecolor}{rgb}{0.969, 0.969, 0.969}\color{fgcolor}\begin{kframe}
\begin{alltt}
\hlkwd{summary}\hlstd{(lasso_regression_result)}
\end{alltt}
\begin{verbatim}
## 
## Call:
## fastcpd(data = data, beta = beta, k = function(x) 0, family = "lasso", 
##     epsilon = 1e-04)
## 
## [1] "Residuals:"
##       Min        1Q    Median        3Q       Max 
## -7.348485 -1.209673  0.094231  1.271128  6.672489 
## 
## Change points:
## [1]  251  497  752 1000 1249
\end{verbatim}
\begin{alltt}
\hlkwd{plot}\hlstd{(lasso_regression_result)}
\end{alltt}
\end{kframe}
\includegraphics[width=\maxwidth]{figure/lasso_regression_plot-1} 
\end{knitrout}

\begin{table}[t!]
  \centering
  \begin{tabular}{@{}cccccccccc@{}}
  \toprule
                         & \multicolumn{3}{c}{s=3}           & \multicolumn{3}{c}{s=6}           & \multicolumn{3}{c}{s=10}          \\ \midrule
  \multirow{4}{*}{cpc=0} & 1.00, 65.24s & 1.00, 35.13s & 1.00, 29.76s \\
                         & 0.89, 2496.32s & 1.00, 4380.03s & 1.00, 1849.64s \\
                         & 1.00, 4922.51s & 1.00, 3153.90s & 1.00, 2737.14s \\
                         & 0.00, 0.00s & 0.00, 0.00s & 0.00, 0.00s \\ \midrule
  \multirow{4}{*}{cpc=1} & 1.00, 48.68s & 0.99, 32.39s & 0.99, 33.05s \\
                         & 0.98, 2571.30s & 0.99, 3523.90s & 1.00, 2049.49s \\
                         & 0.98, 3551.50s & 1.00, 2679.05s & 0.99, 2218.52s \\
                         & 0.00, 0.00s & 0.00, 0.00s & 0.00, 0.00s \\ \midrule
  \multirow{4}{*}{cpc=3} & 0.96, 53.23s & 1.00, 40.23s & 0.99, 32.94s \\
                         & 0.98, 1720.88s & 0.99, 1187.46s & 1.00, 858.13s \\
                         & 0.98, 2608.36s & 0.99, 2164.48s & 1.00, 1440.42s \\
                         & 0.00, 0.00s & 0.00, 0.00s & 0.00, 0.00s \\ \midrule
  \multirow{4}{*}{cpc=5} & 1.00, 33.08s & 0.99, 26.20s & 1.00, 21.56s \\
                         & 0.97, 1152.88s & 1.00, 852.29s & 0.99, 587.59s \\
                         & 0.98, 1492.22s & 1.00, 1267.07s & 1.00, 698.19s \\
                         & 0.00, 0.00s & 0.00, 0.00s & 0.00, 0.00s \\ \bottomrule
  \end{tabular}
  \caption{\label{tab:poisson regression comparison} Comparison of the algorithms for the poisson regression model in mean.}
\end{table}

\section{Changes in user specified models with custom cost functions} \label{sec:custom}

\subsection{Reproducing built-in models} \label{sec:reproduce}



\begin{knitrout}
\definecolor{shadecolor}{rgb}{0.969, 0.969, 0.969}\color{fgcolor}\begin{kframe}
\begin{alltt}
\hlstd{logistic_regression_result} \hlkwb{<-} \hlkwd{fastcpd}\hlstd{(}
  \hlkwc{data} \hlstd{= data,}
  \hlkwc{beta} \hlstd{= beta,}
  \hlkwc{family} \hlstd{=} \hlstr{"binomial"}\hlstd{,}
  \hlkwc{k} \hlstd{=} \hlkwa{function}\hlstd{(}\hlkwc{x}\hlstd{)} \hlnum{20}\hlstd{,}
  \hlkwc{epsilon} \hlstd{=} \hlnum{1e-5}
\hlstd{)}
\hlstd{logistic_loss} \hlkwb{<-} \hlkwa{function}\hlstd{(}\hlkwc{data}\hlstd{,} \hlkwc{theta}\hlstd{) \{}
  \hlstd{u} \hlkwb{<-} \hlkwd{c}\hlstd{(data[,} \hlopt{-}\hlnum{1}\hlstd{,} \hlkwc{drop} \hlstd{=} \hlnum{FALSE}\hlstd{]} \hlopt \hlstd{theta)}
  \hlkwd{sum}\hlstd{(}\hlopt{-}\hlstd{data[,} \hlnum{1}\hlstd{]} \hlopt{*} \hlstd{u} \hlopt{+} \hlkwd{log}\hlstd{(}\hlnum{1} \hlopt{+} \hlkwd{exp}\hlstd{(u)))}
\hlstd{\}}

\hlstd{logistic_loss_gradient} \hlkwb{<-} \hlkwa{function}\hlstd{(}\hlkwc{data}\hlstd{,} \hlkwc{theta}\hlstd{) \{}
  \hlstd{x} \hlkwb{<-} \hlstd{data[}\hlopt{-}\hlnum{1}\hlstd{]}
  \hlstd{y} \hlkwb{<-} \hlstd{data[}\hlnum{1}\hlstd{]}
  \hlkwd{c}\hlstd{(}\hlopt{-}\hlstd{(y} \hlopt{-} \hlnum{1} \hlopt{/} \hlstd{(}\hlnum{1} \hlopt{+} \hlkwd{exp}\hlstd{(}\hlopt{-}\hlstd{x} \hlopt \hlstd{theta))))} \hlopt{*} \hlstd{x}
\hlstd{\}}

\hlstd{logistic_loss_hessian} \hlkwb{<-} \hlkwa{function}\hlstd{(}\hlkwc{data}\hlstd{,} \hlkwc{theta}\hlstd{,} \hlkwc{hessian}\hlstd{) \{}
  \hlstd{data_x} \hlkwb{<-} \hlstd{data[}\hlopt{-}\hlnum{1}\hlstd{]}
  \hlstd{prob} \hlkwb{<-} \hlnum{1} \hlopt{/} \hlstd{(}\hlnum{1} \hlopt{+} \hlkwd{exp}\hlstd{(}\hlopt{-}\hlstd{data_x} \hlopt \hlstd{theta))}
  \hlstd{hessian} \hlopt{+} \hlstd{(data_x} \hlopt \hlstd{data_x)} \hlopt{*} \hlkwd{c}\hlstd{((}\hlnum{1} \hlopt{-} \hlstd{prob)} \hlopt{*} \hlstd{prob)}
\hlstd{\}}
\hlstd{logistic_regression_result_custom} \hlkwb{<-} \hlkwd{fastcpd}\hlstd{(}
  \hlkwc{data} \hlstd{= data,}
  \hlkwc{beta} \hlstd{= beta,}
  \hlkwc{k} \hlstd{=} \hlkwa{function}\hlstd{(}\hlkwc{x}\hlstd{)} \hlnum{20}\hlstd{,}
  \hlkwc{epsilon} \hlstd{=} \hlnum{1e-5}\hlstd{,}
  \hlkwc{cost} \hlstd{= logistic_loss,}
  \hlkwc{cost_gradient} \hlstd{= logistic_loss_gradient,}
  \hlkwc{cost_hessian} \hlstd{= logistic_loss_hessian}
\hlstd{)}
\hlstd{logistic_regression_result}\hlopt{@}\hlkwc{cp_set}
\end{alltt}
\begin{verbatim}
## [1]   64  301  681 1130 1444
\end{verbatim}
\begin{alltt}
\hlstd{logistic_regression_result_custom}\hlopt{@}\hlkwc{cp_set}
\end{alltt}
\begin{verbatim}
## [1]  291  685 1163
\end{verbatim}
\end{kframe}
\end{knitrout}

\subsection{Changes in Huber regression models} \label{sec:huber}

\begin{knitrout}
\definecolor{shadecolor}{rgb}{0.969, 0.969, 0.969}\color{fgcolor}\begin{kframe}
\begin{alltt}
\hlstd{n} \hlkwb{<-} \hlnum{400} \hlopt{+} \hlnum{300} \hlopt{+} \hlnum{400}
\hlstd{p} \hlkwb{<-} \hlnum{3}
\hlstd{x} \hlkwb{<-} \hlstd{mvtnorm}\hlopt{::}\hlkwd{rmvnorm}\hlstd{(n,} \hlkwc{mean} \hlstd{=} \hlkwd{rep}\hlstd{(}\hlnum{0}\hlstd{, p),} \hlkwc{sigma} \hlstd{=} \hlkwd{diag}\hlstd{(p))}
\hlstd{theta} \hlkwb{<-} \hlkwd{rbind}\hlstd{(}
  \hlstd{mvtnorm}\hlopt{::}\hlkwd{rmvnorm}\hlstd{(}\hlnum{1}\hlstd{,} \hlkwc{mean} \hlstd{=} \hlkwd{rep}\hlstd{(}\hlnum{0}\hlstd{, p),} \hlkwc{sigma} \hlstd{=} \hlkwd{diag}\hlstd{(p)),}
  \hlstd{mvtnorm}\hlopt{::}\hlkwd{rmvnorm}\hlstd{(}\hlnum{1}\hlstd{,} \hlkwc{mean} \hlstd{=} \hlkwd{rep}\hlstd{(}\hlnum{3}\hlstd{, p),} \hlkwc{sigma} \hlstd{=} \hlkwd{diag}\hlstd{(p)),}
  \hlstd{mvtnorm}\hlopt{::}\hlkwd{rmvnorm}\hlstd{(}\hlnum{1}\hlstd{,} \hlkwc{mean} \hlstd{=} \hlkwd{rep}\hlstd{(}\hlnum{5}\hlstd{, p),} \hlkwc{sigma} \hlstd{=} \hlkwd{diag}\hlstd{(p))}
\hlstd{)}
\hlstd{theta} \hlkwb{<-} \hlstd{theta[}\hlkwd{rep}\hlstd{(}\hlkwd{seq_len}\hlstd{(}\hlnum{3}\hlstd{),} \hlkwd{c}\hlstd{(}\hlnum{400}\hlstd{,} \hlnum{300}\hlstd{,} \hlnum{400}\hlstd{)), ]}
\hlstd{y_true} \hlkwb{<-} \hlkwd{rowSums}\hlstd{(x} \hlopt{*} \hlstd{theta)}
\hlstd{factor} \hlkwb{<-} \hlkwd{c}\hlstd{(}
  \hlnum{2} \hlopt{*} \hlstd{stats}\hlopt{::}\hlkwd{rbinom}\hlstd{(}\hlnum{400}\hlstd{,} \hlkwc{size} \hlstd{=} \hlnum{1}\hlstd{,} \hlkwc{prob} \hlstd{=} \hlnum{0.95}\hlstd{)} \hlopt{-} \hlnum{1}\hlstd{,}
  \hlnum{2} \hlopt{*} \hlstd{stats}\hlopt{::}\hlkwd{rbinom}\hlstd{(}\hlnum{300}\hlstd{,} \hlkwc{size} \hlstd{=} \hlnum{1}\hlstd{,} \hlkwc{prob} \hlstd{=} \hlnum{0.95}\hlstd{)} \hlopt{-} \hlnum{1}\hlstd{,}
  \hlnum{2} \hlopt{*} \hlstd{stats}\hlopt{::}\hlkwd{rbinom}\hlstd{(}\hlnum{400}\hlstd{,} \hlkwc{size} \hlstd{=} \hlnum{1}\hlstd{,} \hlkwc{prob} \hlstd{=} \hlnum{0.95}\hlstd{)} \hlopt{-} \hlnum{1}
\hlstd{)}
\hlstd{y} \hlkwb{<-} \hlstd{factor} \hlopt{*} \hlstd{y_true} \hlopt{+} \hlstd{stats}\hlopt{::}\hlkwd{rnorm}\hlstd{(n)}
\hlstd{data} \hlkwb{<-} \hlkwd{cbind}\hlstd{(y, x)}

\hlstd{huber_loss} \hlkwb{<-} \hlkwa{function}\hlstd{(}
  \hlkwc{data}\hlstd{,}
  \hlkwc{theta}\hlstd{,}
  \hlkwc{threshold} \hlstd{=} \hlnum{1}
\hlstd{) \{}
  \hlstd{residual} \hlkwb{<-} \hlstd{data[,} \hlnum{1}\hlstd{]} \hlopt{-} \hlstd{data[,} \hlopt{-}\hlnum{1}\hlstd{,} \hlkwc{drop} \hlstd{=} \hlnum{FALSE}\hlstd{]} \hlopt \hlstd{theta}
  \hlstd{indicator} \hlkwb{<-} \hlkwd{abs}\hlstd{(residual)} \hlopt{<=} \hlstd{threshold}
  \hlkwd{sum}\hlstd{(residual} \hlopt{^} \hlnum{2} \hlopt{/} \hlnum{2} \hlopt{*} \hlstd{indicator} \hlopt{+} \hlstd{threshold} \hlopt{*} \hlstd{(}\hlkwd{abs}\hlstd{(residual)} \hlopt{-} \hlstd{threshold} \hlopt{/} \hlnum{2}\hlstd{)} \hlopt{*} \hlstd{(}\hlnum{1} \hlopt{-} \hlstd{indicator))}
\hlstd{\}}

\hlstd{huber_loss_gradient} \hlkwb{<-} \hlkwa{function}\hlstd{(}\hlkwc{data}\hlstd{,} \hlkwc{theta}\hlstd{,} \hlkwc{threshold} \hlstd{=} \hlnum{1}\hlstd{) \{}
  \hlstd{residual} \hlkwb{<-} \hlkwd{c}\hlstd{(data[}\hlnum{1}\hlstd{]} \hlopt{-} \hlstd{data[}\hlopt{-}\hlnum{1}\hlstd{]} \hlopt \hlstd{theta)}
  \hlkwa{if} \hlstd{(}\hlkwd{abs}\hlstd{(residual)} \hlopt{<=} \hlstd{threshold) \{}
    \hlopt{-} \hlstd{residual} \hlopt{*} \hlstd{data[}\hlopt{-}\hlnum{1}\hlstd{]}
  \hlstd{\}} \hlkwa{else} \hlstd{\{}
    \hlopt{-} \hlstd{threshold} \hlopt{*} \hlkwd{sign}\hlstd{(residual)} \hlopt{*} \hlstd{data[}\hlopt{-}\hlnum{1}\hlstd{]}
  \hlstd{\}}
\hlstd{\}}

\hlstd{huber_loss_hessian} \hlkwb{<-} \hlkwa{function}\hlstd{(}
  \hlkwc{data}\hlstd{,}
  \hlkwc{theta}\hlstd{,}
  \hlkwc{hessian}\hlstd{,}
  \hlkwc{threshold} \hlstd{=} \hlnum{1}
\hlstd{) \{}
  \hlstd{residual} \hlkwb{<-} \hlkwd{c}\hlstd{(data[}\hlnum{1}\hlstd{]} \hlopt{-} \hlstd{data[}\hlopt{-}\hlnum{1}\hlstd{]} \hlopt \hlstd{theta)}
  \hlkwa{if} \hlstd{(}\hlkwd{abs}\hlstd{(residual)} \hlopt{<=} \hlstd{threshold) \{}
    \hlstd{hessian} \hlopt{+} \hlkwd{outer}\hlstd{(data[}\hlopt{-}\hlnum{1}\hlstd{], data[}\hlopt{-}\hlnum{1}\hlstd{])}
  \hlstd{\}} \hlkwa{else} \hlstd{\{}
    \hlstd{hessian} \hlopt{+} \hlnum{0.01} \hlopt{*} \hlkwd{diag}\hlstd{(}\hlkwd{length}\hlstd{(theta))}
  \hlstd{\}}
\hlstd{\}}

\hlstd{huber_regression_result} \hlkwb{<-} \hlkwd{fastcpd}\hlstd{(}
  \hlkwc{data} \hlstd{= data,}
  \hlkwc{beta} \hlstd{= (p} \hlopt{+} \hlnum{1}\hlstd{)} \hlopt{*} \hlkwd{log}\hlstd{(n),}
  \hlkwc{cost} \hlstd{= huber_loss,}
  \hlkwc{cost_gradient} \hlstd{= huber_loss_gradient,}
  \hlkwc{cost_hessian} \hlstd{= huber_loss_hessian}
\hlstd{)}

\hlkwd{summary}\hlstd{(huber_regression_result)}
\end{alltt}
\begin{verbatim}
## 
## Call:
## fastcpd(data = data, beta = (p + 1) * log(n), cost = huber_loss, 
##     cost_gradient = huber_loss_gradient, cost_hessian = huber_loss_hessian)
## 
## Change points:
## [1] 396 715
\end{verbatim}
\end{kframe}
\end{knitrout}

\subsection{Changes in quantile regression models} \label{sec:quantile}

%
\begin{knitrout}
\definecolor{shadecolor}{rgb}{0.969, 0.969, 0.969}\color{fgcolor}\begin{kframe}
\begin{alltt}
\hlstd{n} \hlkwb{<-} \hlnum{450}
\hlstd{p} \hlkwb{<-} \hlnum{3}
\hlstd{data} \hlkwb{<-} \hlstd{mvtnorm}\hlopt{::}\hlkwd{rmvnorm}\hlstd{(n} \hlopt{/} \hlnum{5}\hlstd{,} \hlkwc{mean} \hlstd{=} \hlkwd{rep}\hlstd{(}\hlnum{0}\hlstd{, p),} \hlkwc{sigma} \hlstd{=} \hlkwd{diag}\hlstd{(p))[}\hlkwd{rep}\hlstd{(}\hlnum{1}\hlopt{:}\hlstd{(n} \hlopt{/} \hlnum{5}\hlstd{),} \hlkwc{each} \hlstd{=} \hlnum{5}\hlstd{), ]}
\hlstd{theta} \hlkwb{<-} \hlkwd{matrix}\hlstd{(}\hlkwd{rnorm}\hlstd{(p} \hlopt{*} \hlnum{2}\hlstd{),} \hlnum{2}\hlstd{, p)[}\hlkwd{rep}\hlstd{(}\hlnum{1}\hlopt{:}\hlnum{2}\hlstd{,} \hlkwd{c}\hlstd{(n} \hlopt{*} \hlnum{2} \hlopt{/} \hlnum{3}\hlstd{, n} \hlopt{/} \hlnum{3}\hlstd{)), ]}
\hlstd{xb} \hlkwb{<-} \hlkwd{rowSums}\hlstd{(data} \hlopt{*} \hlstd{theta)}
\hlstd{xb[}\hlnum{1}\hlopt{:}\hlstd{(n} \hlopt{/} \hlnum{3} \hlopt{/} \hlnum{5}\hlstd{)} \hlopt{*} \hlnum{5} \hlopt{+} \hlstd{(n} \hlopt{/} \hlnum{3}\hlstd{)} \hlopt{-} \hlnum{4}\hlstd{]} \hlkwb{<-} \hlstd{xb[}\hlnum{1}\hlopt{:}\hlstd{(n} \hlopt{/} \hlnum{3} \hlopt{/} \hlnum{5}\hlstd{)} \hlopt{*} \hlnum{5} \hlopt{+} \hlstd{(n} \hlopt{/} \hlnum{3}\hlstd{)} \hlopt{-} \hlnum{4}\hlstd{]} \hlopt{+} \hlnum{50}
\hlstd{xb[}\hlnum{1}\hlopt{:}\hlstd{(n} \hlopt{/} \hlnum{3} \hlopt{/} \hlnum{5}\hlstd{)} \hlopt{*} \hlnum{5} \hlopt{+} \hlstd{(n} \hlopt{/} \hlnum{3}\hlstd{)} \hlopt{-} \hlnum{3}\hlstd{]} \hlkwb{<-} \hlstd{xb[}\hlnum{1}\hlopt{:}\hlstd{(n} \hlopt{/} \hlnum{3} \hlopt{/} \hlnum{5}\hlstd{)} \hlopt{*} \hlnum{5} \hlopt{+} \hlstd{(n} \hlopt{/} \hlnum{3}\hlstd{)} \hlopt{-} \hlnum{3}\hlstd{]} \hlopt{+} \hlnum{40}
\hlstd{xb[}\hlnum{1}\hlopt{:}\hlstd{(n} \hlopt{/} \hlnum{3} \hlopt{/} \hlnum{5}\hlstd{)} \hlopt{*} \hlnum{5} \hlopt{+} \hlstd{(n} \hlopt{/} \hlnum{3}\hlstd{)} \hlopt{-} \hlnum{1}\hlstd{]} \hlkwb{<-} \hlstd{xb[}\hlnum{1}\hlopt{:}\hlstd{(n} \hlopt{/} \hlnum{3} \hlopt{/} \hlnum{5}\hlstd{)} \hlopt{*} \hlnum{5} \hlopt{+} \hlstd{(n} \hlopt{/} \hlnum{3}\hlstd{)} \hlopt{-} \hlnum{1}\hlstd{]} \hlopt{-} \hlnum{5}
\hlstd{xb[}\hlnum{1}\hlopt{:}\hlstd{(n} \hlopt{/} \hlnum{3} \hlopt{/} \hlnum{5}\hlstd{)} \hlopt{*} \hlnum{5} \hlopt{+} \hlstd{(n} \hlopt{/} \hlnum{3}\hlstd{)]} \hlkwb{<-} \hlstd{xb[}\hlnum{1}\hlopt{:}\hlstd{(n} \hlopt{/} \hlnum{3} \hlopt{/} \hlnum{5}\hlstd{)} \hlopt{*} \hlnum{5} \hlopt{+} \hlstd{(n} \hlopt{/} \hlnum{3}\hlstd{)]} \hlopt{-} \hlnum{10}
\hlstd{data} \hlkwb{<-} \hlkwd{cbind}\hlstd{(xb, data)}
\end{alltt}
\end{kframe}
\end{knitrout}
%
Verify that the data contains three parts with two segments. The first segment
is from 1 to 300 and the second segment is from 301 to 450. By doing piecewise
quantile regression and linear regression, we can verify that coefficients of
quantile regression in the first 150 observations are the same as those in the
second 150 observations. The coefficients of linear regression in the first 150
observations are different from those in the second 150 observations.

\begin{table}[t!]
\centering
\begin{tabular}{llp{10cm}}
\hline
Algorithm           & Data    & Coefficients \\ \hline
Quantile Regression & 1st 150 & -1.3890396, 1.8959482, -0.5951861 \\
Quantile Regression & 2nd 150 & -1.3890396, 1.8959482, -0.5951861 \\
Quantile Regression & 3rd 150 & 0.4124297, -0.7807476, -1.481476 \\
Linear Regression   & 1st 150 & -1.3890396, 1.8959482, -0.5951861 \\ \hline
Linear Regression   & 2nd 150 & -2.5457644, 2.9560383, 0.5427844 \\ \hline
Linear Regression   & 3rd 150 & 0.4124297, -0.7807476, -1.481476 \\ \hline
\end{tabular}
\caption{\label{tab:logistic regression comparison} Comparison of the
algorithms for the logistic regression model.}
\end{table}

Let see what is the output if we misspecify the model.
%
\begin{knitrout}
\definecolor{shadecolor}{rgb}{0.969, 0.969, 0.969}\color{fgcolor}\begin{kframe}
\begin{alltt}
\hlcom{# linear_regression_result <- fastcpd(}
\hlcom{#   data = data,}
\hlcom{#   beta = log(n) / 2,}
\hlcom{#   segment_count = 10,}
\hlcom{#   trim = 0.025,}
\hlcom{#   k = function(x) 0,}
\hlcom{#   family = "gaussian",}
\hlcom{#   epsilon = 1e-5}
\hlcom{# )}

\hlcom{# summary(linear_regression_result)}
\end{alltt}
\end{kframe}
\end{knitrout}
%
The output is not correct. Now let see what is the output if we specify the
model to be a quantile regression model using the custom cost function.
%
\begin{knitrout}
\definecolor{shadecolor}{rgb}{0.969, 0.969, 0.969}\color{fgcolor}\begin{kframe}
\begin{alltt}
\hlstd{quantile_loss} \hlkwb{<-} \hlkwa{function}\hlstd{(}
  \hlkwc{data}\hlstd{,}
  \hlkwc{theta}\hlstd{,}
  \hlkwc{quant} \hlstd{=} \hlnum{0.5}\hlstd{,}
  \hlkwc{smoothing} \hlstd{=} \hlnum{0.25}
\hlstd{) \{}
  \hlstd{residual} \hlkwb{<-} \hlstd{data[,} \hlnum{1}\hlstd{]} \hlopt{-} \hlstd{data[,} \hlopt{-}\hlnum{1}\hlstd{,} \hlkwc{drop} \hlstd{=} \hlnum{FALSE}\hlstd{]} \hlopt \hlstd{theta}
  \hlkwd{mean}\hlstd{(residual} \hlopt{*} \hlstd{(quant} \hlopt{-} \hlstd{(residual} \hlopt{<} \hlnum{0}\hlstd{)))}
\hlstd{\}}

\hlstd{quantile_loss_gradient} \hlkwb{<-} \hlkwa{function}\hlstd{(}
  \hlkwc{data}\hlstd{,} \hlkwc{theta}\hlstd{,} \hlkwc{quant} \hlstd{=} \hlnum{0.5}\hlstd{,} \hlkwc{smoothing} \hlstd{=} \hlnum{0.25}
\hlstd{) \{}
  \hlkwd{c}\hlstd{(quant} \hlopt{-} \hlnum{1} \hlopt{/} \hlstd{(}\hlnum{1} \hlopt{+} \hlkwd{exp}\hlstd{(}\hlkwd{c}\hlstd{(data[}\hlnum{1}\hlstd{]} \hlopt{-} \hlstd{data[}\hlopt{-}\hlnum{1}\hlstd{]} \hlopt \hlstd{theta)} \hlopt{/} \hlstd{smoothing)))} \hlopt{*} \hlstd{data[}\hlopt{-}\hlnum{1}\hlstd{]}
\hlstd{\}}

\hlstd{quantile_loss_hessian} \hlkwb{<-} \hlkwa{function}\hlstd{(}
  \hlkwc{data}\hlstd{,}
  \hlkwc{theta}\hlstd{,}
  \hlkwc{hessian}\hlstd{,}
  \hlkwc{quant} \hlstd{=} \hlnum{0.5}\hlstd{,}
  \hlkwc{smoothing} \hlstd{=} \hlnum{0.25}
\hlstd{) \{}
  \hlcom{# hessian - exp(c(data[1] - data[-1] %*% theta) / smoothing) / (smoothing * (1 + exp(c(data[1] - data[-1] %*% theta) / smoothing))^2) * data[-1] %o% data[-1]}
  \hlstd{hessian} \hlopt{+} \hlnum{0.01} \hlopt{*} \hlkwd{diag}\hlstd{(}\hlkwd{length}\hlstd{(theta))}
\hlstd{\}}

\hlcom{# print(fastcpd(}
\hlcom{#   data = data,}
\hlcom{#   beta = log(n) / 2,}
\hlcom{#   segment_count = 5,}
\hlcom{#   trim = 0.025,}
\hlcom{#   k = function(x) 0,}
\hlcom{#   cost = quantile_loss,}
\hlcom{#   cost_gradient = quantile_loss_gradient,}
\hlcom{#   cost_hessian = quantile_loss_hessian}
\hlcom{# ))}

\hlcom{# print(fastcpd(}
\hlcom{#   data = data,}
\hlcom{#   beta = log(n) / 2,}
\hlcom{#   segment_count = 5,}
\hlcom{#   trim = 0.025,}
\hlcom{#   k = function(x) 4,}
\hlcom{#   cost = quantile_loss,}
\hlcom{#   cost_gradient = quantile_loss_gradient,}
\hlcom{#   cost_hessian = quantile_loss_hessian}
\hlcom{# ))}

\hlcom{# print(fastcpd(}
\hlcom{#   data = data,}
\hlcom{#   beta = log(n) / 2,}
\hlcom{#   segment_count = 5,}
\hlcom{#   trim = 0.025,}
\hlcom{#   k = function(x) 9,}
\hlcom{#   cost = quantile_loss,}
\hlcom{#   cost_gradient = quantile_loss_gradient,}
\hlcom{#   cost_hessian = quantile_loss_hessian}
\hlcom{# ))}

\hlcom{# benchmarked <- microbenchmark::microbenchmark(}
\hlcom{#   "k = 1" = fastcpd(}
\hlcom{#     data = data,}
\hlcom{#     beta = log(n) / 2,}
\hlcom{#     segment_count = 5,}
\hlcom{#     trim = 0.025,}
\hlcom{#     k = function(x) 0,}
\hlcom{#     cost = quantile_loss,}
\hlcom{#     cost_gradient = quantile_loss_gradient,}
\hlcom{#     cost_hessian = quantile_loss_hessian}
\hlcom{#   ),}
\hlcom{#   "k = 5" = fastcpd(}
\hlcom{#     data = data,}
\hlcom{#     beta = log(n) / 2,}
\hlcom{#     segment_count = 5,}
\hlcom{#     trim = 0.025,}
\hlcom{#     k = function(x) 4,}
\hlcom{#     cost = quantile_loss,}
\hlcom{#     cost_gradient = quantile_loss_gradient,}
\hlcom{#     cost_hessian = quantile_loss_hessian}
\hlcom{#   ),}
\hlcom{#   "k = 10" = fastcpd(}
\hlcom{#     data = data,}
\hlcom{#     beta = log(n) / 2,}
\hlcom{#     segment_count = 5,}
\hlcom{#     trim = 0.025,}
\hlcom{#     k = function(x) 9,}
\hlcom{#     cost = quantile_loss,}
\hlcom{#     cost_gradient = quantile_loss_gradient,}
\hlcom{#     cost_hessian = quantile_loss_hessian}
\hlcom{#   ),}
\hlcom{#   times = 2}
\hlcom{# )}
\hlcom{# print(benchmarked)}
\end{alltt}
\end{kframe}
\end{knitrout}
%

\begin{figure}[t!]
\centering
% <<quantile regression visualization, echo=FALSE, fig=TRUE, height=5.2, width=7>>=
% ggplot2::autoplot(benchmarked)
% @
\caption{\label{fig:quantile regression bench mark} Benchmarking results for
the quantile regression model.}
\end{figure}

\subsection{Changes in mean shift models} \label{sec:mean shift model}

Now let's consider the mean shift model using custom cost function. Since a
mean shift model is a special case of a linear regression model, by specifying a
response variable we can use the linear regression model to detect the change
points.
%
\begin{knitrout}
\definecolor{shadecolor}{rgb}{0.969, 0.969, 0.969}\color{fgcolor}\begin{kframe}
\begin{alltt}
\hlstd{p} \hlkwb{<-} \hlnum{1}
\hlstd{data} \hlkwb{<-} \hlkwd{rbind}\hlstd{(}
  \hlstd{mvtnorm}\hlopt{::}\hlkwd{rmvnorm}\hlstd{(}\hlnum{100}\hlstd{,} \hlkwc{mean} \hlstd{=} \hlkwd{rep}\hlstd{(}\hlnum{0}\hlstd{, p),} \hlkwc{sigma} \hlstd{=} \hlkwd{diag}\hlstd{(}\hlnum{1}\hlstd{, p)),}
  \hlstd{mvtnorm}\hlopt{::}\hlkwd{rmvnorm}\hlstd{(}\hlnum{100}\hlstd{,} \hlkwc{mean} \hlstd{=} \hlkwd{rep}\hlstd{(}\hlnum{5}\hlstd{, p),} \hlkwc{sigma} \hlstd{=} \hlkwd{diag}\hlstd{(}\hlnum{1}\hlstd{, p))}
\hlstd{)}
\hlstd{data} \hlkwb{<-} \hlkwd{cbind}\hlstd{(}\hlkwd{c}\hlstd{(data[}\hlnum{1}\hlopt{:}\hlnum{100}\hlstd{, ,} \hlkwc{drop} \hlstd{=} \hlnum{FALSE}\hlstd{]} \hlopt \hlkwd{colMeans}\hlstd{(data[}\hlnum{1}\hlopt{:}\hlnum{100}\hlstd{, ,} \hlkwc{drop} \hlstd{=} \hlnum{FALSE}\hlstd{]), data[}\hlnum{101}\hlopt{:}\hlnum{200}\hlstd{, ,} \hlkwc{drop} \hlstd{=} \hlnum{FALSE}\hlstd{]} \hlopt \hlkwd{colMeans}\hlstd{(data[}\hlnum{101}\hlopt{:}\hlnum{200}\hlstd{, ,} \hlkwc{drop} \hlstd{=} \hlnum{FALSE}\hlstd{])), data)}

\hlstd{mean_shift_linear_result} \hlkwb{<-} \hlkwd{fastcpd}\hlstd{(}
  \hlkwc{data} \hlstd{= data,}
  \hlkwc{beta} \hlstd{=} \hlkwd{log}\hlstd{(}\hlnum{200}\hlstd{)} \hlopt{/} \hlnum{2}\hlstd{,}
  \hlkwc{segment_count} \hlstd{=} \hlnum{3}\hlstd{,}
  \hlkwc{trim} \hlstd{=} \hlnum{0.025}\hlstd{,}
  \hlkwc{k} \hlstd{=} \hlkwa{function}\hlstd{(}\hlkwc{x}\hlstd{)} \hlnum{0}\hlstd{,}
  \hlkwc{family} \hlstd{=} \hlstr{"gaussian"}\hlstd{,}
  \hlkwc{epsilon} \hlstd{=} \hlnum{1e-5}
\hlstd{)}
\hlkwd{summary}\hlstd{(mean_shift_linear_result)}
\end{alltt}
\begin{verbatim}
## 
## Call:
## fastcpd(data = data, beta = log(200)/2, segment_count = 3, trim = 0.025, 
##     k = function(x) 0, family = "gaussian", epsilon = 1e-05)
## 
## [1] "Residuals:"
##           Min            1Q        Median            3Q           Max 
## -1.387779e-17 -4.336810e-19  0.000000e+00  0.000000e+00  2.775558e-17 
## 
## Change points:
## [1] 100
\end{verbatim}
\end{kframe}
\end{knitrout}
%
Now let's see what would be the output if we use the custom cost function.
%
\begin{knitrout}
\definecolor{shadecolor}{rgb}{0.969, 0.969, 0.969}\color{fgcolor}\begin{kframe}
\begin{alltt}
\hlstd{data} \hlkwb{<-} \hlstd{data[,} \hlopt{-}\hlnum{1}\hlstd{,} \hlkwc{drop} \hlstd{=} \hlnum{FALSE}\hlstd{]}

\hlstd{mean_loss} \hlkwb{<-} \hlkwa{function}\hlstd{(}\hlkwc{data}\hlstd{,} \hlkwc{theta}\hlstd{) \{}
  \hlkwd{norm}\hlstd{(}\hlkwd{sweep}\hlstd{(data,} \hlnum{2}\hlstd{, theta),} \hlkwc{type} \hlstd{=} \hlstr{"F"}\hlstd{)} \hlopt{^} \hlnum{2} \hlopt{/} \hlnum{2}
\hlstd{\}}

\hlstd{mean_loss_gradient} \hlkwb{<-} \hlkwa{function}\hlstd{(}\hlkwc{data}\hlstd{,} \hlkwc{theta}\hlstd{) \{}
  \hlstd{theta} \hlopt{-} \hlstd{data}
\hlstd{\}}

\hlstd{mean_loss_hessian} \hlkwb{<-} \hlkwa{function}\hlstd{(}\hlkwc{data}\hlstd{,} \hlkwc{theta}\hlstd{,} \hlkwc{hessian}\hlstd{) \{}
  \hlstd{hessian} \hlopt{+} \hlkwd{diag}\hlstd{(p)}
\hlstd{\}}

\hlstd{mean_loss_result} \hlkwb{<-} \hlkwd{fastcpd}\hlstd{(}
  \hlkwc{data} \hlstd{= data,}
  \hlkwc{beta} \hlstd{=} \hlnum{10}\hlstd{,}
  \hlkwc{segment_count} \hlstd{=} \hlnum{4}\hlstd{,}
  \hlkwc{trim} \hlstd{=} \hlnum{0.025}\hlstd{,}
  \hlkwc{k} \hlstd{=} \hlkwa{function}\hlstd{(}\hlkwc{x}\hlstd{)} \hlnum{0}\hlstd{,}
  \hlkwc{p} \hlstd{= p,}
  \hlkwc{cost} \hlstd{= mean_loss,}
  \hlkwc{cost_gradient} \hlstd{= mean_loss_gradient,}
  \hlkwc{cost_hessian} \hlstd{= mean_loss_hessian}
\hlstd{)}

\hlkwd{summary}\hlstd{(mean_loss_result)}
\end{alltt}
\begin{verbatim}
## 
## Call:
## fastcpd(data = data, beta = 10, segment_count = 4, trim = 0.025, 
##     k = function(x) 0, p = p, cost = mean_loss, cost_gradient = mean_loss_gradient, 
##     cost_hessian = mean_loss_hessian)
## 
## Change points:
## [1] 100
\end{verbatim}
\end{kframe}
\end{knitrout}

\section{Adaptive number of epochs} \label{sec:adaptive number of epochs}

In this section, we will discuss how to use the adaptive number of epochs
feature in \pkg{fastcpd}. The adaptive number of epochs feature is designed to
automatically determine the number of epochs to run the algorithm. The criteria
for determining the number of epochs is based on the following function:
%
\begin{equation} \label{eq:adaptive number of epochs}
  K = \min(\max(\boldsymbol{k}, \lceil \frac{1000}{\text{segment length}} \rceil), 10)
\end{equation}
%
where $K$ is the number of epochs to run the algorithm, $\boldsymbol{k}$ is the
number of epochs specified by the user parameter \code{sgd_k}, and \code{segment length} is the
length of the segment. The goal is to upper limit the number of epochs to 10
when the segment length is small and to lower limit the number of epochs to
$\boldsymbol{k}$. The transition between the upper limit and lower limit is
calculated by $\lceil 1000 / \text{segment length} \rceil$.

\begin{knitrout}
\definecolor{shadecolor}{rgb}{0.969, 0.969, 0.969}\color{fgcolor}\begin{kframe}
\begin{alltt}
\hlstd{k} \hlkwb{<-} \hlkwa{function}\hlstd{(}\hlkwc{x}\hlstd{,} \hlkwc{n} \hlstd{=} \hlnum{1500}\hlstd{) \{}
  \hlkwa{if} \hlstd{(x} \hlopt{<} \hlstd{n} \hlopt{/} \hlnum{10} \hlopt{*} \hlnum{1} \hlopt{/} \hlnum{3}\hlstd{)} \hlnum{3}
  \hlkwa{else if} \hlstd{(x} \hlopt{<} \hlstd{n} \hlopt{/} \hlnum{10} \hlopt{*} \hlnum{2} \hlopt{/} \hlnum{3}\hlstd{)} \hlnum{2}
  \hlkwa{else if} \hlstd{(x} \hlopt{<} \hlstd{n} \hlopt{/} \hlnum{10}\hlstd{)} \hlnum{1}
  \hlkwa{else} \hlnum{0}
\hlstd{\}}
\end{alltt}
\end{kframe}
\end{knitrout}

%% -- Summary/conclusions/discussion -------------------------------------------

\section{Summary and discussion} \label{sec:summary}

Our package \pkg{fastcpd} provides a fast and flexible implementation of
change point detection algorithms. The package is designed to be easy to use
and to provide a wide range of options for the user. The package is
implemented in \proglang{R} and is available on GitHub.


%% -- Optional special unnumbered sections -------------------------------------

\section*{Computational details}

\begin{leftbar}
If necessary or useful, information about certain computational details
such as version numbers, operating systems, or compilers could be included
in an unnumbered section. Also, auxiliary packages (say, for visualizations,
maps, tables, \dots) that are not cited in the main text can be credited here.
\end{leftbar}

The results in this paper were obtained using
\proglang{R}~4.1.3 with the
\pkg{MASS}~7.3.58.1 package. \proglang{R} itself
and all packages used are available from the Comprehensive
\proglang{R} Archive Network (CRAN) at
\url{https://CRAN.R-project.org/}.


\section*{Acknowledgments}

\begin{leftbar}
All acknowledgments (note the AE spelling) should be collected in this
unnumbered section before the references. It may contain the usual information
about funding and feedback from colleagues/reviewers/etc. Furthermore,
information such as relative contributions of the authors may be added here
(if any).
\end{leftbar}


%% -- Bibliography -------------------------------------------------------------
%% - References need to be provided in a .bib BibTeX database.
%% - All references should be made with \cite, \citet, \citep, \citealp etc.
%%   (and never hard-coded). See the FAQ for details.
%% - JSS-specific markup (\proglang, \pkg, \code) should be used in the .bib.
%% - Titles in the .bib should be in title case.
%% - DOIs should be included where available.

\bibliography{refs}


%% -- Appendix (if any) --------------------------------------------------------
%% - After the bibliography with page break.
%% - With proper section titles and _not_ just "Appendix".

\newpage

\begin{appendix}

\section{More technical details} \label{app:technical}

\begin{leftbar}
Appendices can be included after the bibliography (with a page break). Each
section within the appendix should have a proper section title (rather than
just \emph{Appendix}).

For more technical style details, please check out JSS's style FAQ at
\url{https://www.jstatsoft.org/pages/view/style#frequently-asked-questions}
which includes the following topics:
\begin{itemize}
  \item Title vs.\ sentence case.
  \item Graphics formatting.
  \item Naming conventions.
  \item Turning JSS manuscripts into \proglang{R} package vignettes.
  \item Trouble shooting.
  \item Many other potentially helpful details\dots
\end{itemize}
\end{leftbar}


\section[Using BibTeX]{Using \textsc{Bib}{\TeX}} \label{app:bibtex}

\begin{leftbar}
References need to be provided in a \textsc{Bib}{\TeX} file (\code{.bib}). All
references should be made with \verb|\cite|, \verb|\citet|, \verb|\citep|,
\verb|\citealp| etc.\ (and never hard-coded). This commands yield different
formats of author-year citations and allow to include additional details (e.g.,
pages, chapters, \dots) in brackets. In case you are not familiar with these
commands see the JSS style FAQ for details.

Cleaning up \textsc{Bib}{\TeX} files is a somewhat tedious task -- especially
when acquiring the entries automatically from mixed online sources. However,
it is important that informations are complete and presented in a consistent
style to avoid confusions. JSS requires the following format.
\begin{itemize}
  \item JSS-specific markup (\verb|\proglang|, \verb|\pkg|, \verb|\code|) should
    be used in the references.
  \item Titles should be in title case.
  \item Journal titles should not be abbreviated and in title case.
  \item DOIs should be included where available.
  \item Software should be properly cited as well. For \proglang{R} packages
    \code{citation("pkgname")} typically provides a good starting point.
\end{itemize}
\end{leftbar}

\end{appendix}

%% -----------------------------------------------------------------------------


\end{document}
